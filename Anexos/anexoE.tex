\begin{minted}{rust}
use diesel::prelude::*;

use crate::schema::{posts as post_schema, users as user_schema};
pub type UserColumns = (
    self::user_schema::user_id,
    self::user_schema::name,
    (self::user_schema::email, self::user_schema::username),
);
pub type PostColumns = (
    self::post_schema::post_id,
    self::post_schema::title,
    self::post_schema::likes,
    self::post_schema::dislikes,
    self::post_schema::author_id,
    self::post_schema::content,
    self::post_schema::content_format,
);
# [:: async_trait :: async_trait (? Send)]
pub trait BlobService: ::std::marker::Send + ::std::marker::Sync + 'static {
    /// List the users in the system.
    async fn list_users(
        &self,
        req: ListUsersRequest,
        http_req: ::actix_web::HttpRequest,
    ) -> ::std::result::Result<ListUsersResponse, ::actix_web::error::Error>;
    /// List a number of posts in the system.
    async fn list_posts(
        &self,
        req: ListPostsRequest,
        http_req: ::actix_web::HttpRequest,
    ) -> ::std::result::Result<ListPostsResponse, ::actix_web::error::Error>;
    /// List comments of a given post.
    async fn list_post_comments(
        &self,
        req: ListPostCommentsRequest,
        http_req: ::actix_web::HttpRequest,
    ) -> ::std::result::Result<ListPostCommentsResponse, ::actix_web::error::Error>;
    /// Create a post in the system.
    async fn create_post(
        &self,
        req: CreatePostRequest,
        http_req: ::actix_web::HttpRequest,
    ) -> ::std::result::Result<Post, ::actix_web::error::Error>;
    /// Post a comment.
    async fn post_comment(
        &self,
        req: PostCommentRequest,
        http_req: ::actix_web::HttpRequest,
    ) -> ::std::result::Result<self::post::Comment, ::actix_web::error::Error>;
    /// {Dis}Like a post.
    async fn like_post(
        &self,
        req: LikePostRequest,
        http_req: ::actix_web::HttpRequest,
    ) -> ::std::result::Result<Post, ::actix_web::error::Error>;
    fn configure(
        self: ::std::sync::Arc<Self>,
        cfg: &mut ::actix_web::web::ServiceConfig)
    where
        Self: ::std::marker::Sized,
    {
        let service: ::std::sync::Arc<dyn BlobService> = self;
        let service: ::actix_web::web::Data<dyn BlobService> = service.into();
        cfg.app_data(service)
            .route(
                "/v1/users",
                actix_web::web::get().to(self::blob_service::list_users),
            )
            .route(
                "/v1/posts",
                actix_web::web::get().to(self::blob_service::list_posts),
            )
            .route(
                "/v1/posts/{post}/comments",
                actix_web::web::get().to(self::blob_service::list_post_comments),
            )
            .route(
                "/v1/posts",
                actix_web::web::post().to(self::blob_service::create_post),
            )
            .route(
                "/v1/posts/{comment.post_id}/comments",
                actix_web::web::post().to(self::blob_service::post_comment),
            )
            .route(
                "/v1/posts/{post}",
                actix_web::web::post().to(self::blob_service::like_post),
            );
    }
}
/// Request for `ListUsers` RPC method.
#[derive(Clone, Debug, Default, PartialEq,
         ::serde::Serialize, ::serde::Deserialize)]
#[serde(rename_all = "camelCase")]
pub struct ListUsersRequest {}
/// Response for `ListUsers` RPC method.
#[derive(Clone, Debug, Default, PartialEq,
         ::serde::Serialize, ::serde::Deserialize)]
#[serde(rename_all = "camelCase")]
pub struct ListUsersResponse {
    /// Users returned for the request.
    #[serde(default)]
    pub users: ::std::vec::Vec<User>,
}
/// Request for `ListPosts` RPC method.
#[derive(Clone, Debug, Default, PartialEq,
         ::serde::Serialize, ::serde::Deserialize)]
#[serde(rename_all = "camelCase")]
pub struct ListPostsRequest {
    /// Return posts after a given ID.
    #[serde(default)]
    pub after: ::std::option::Option<i64>,
    /// Limit the number of returned posts.
    #[serde(default)]
    pub limit: ::std::option::Option<u32>,
}
/// Response for `ListPosts` RPC method
#[derive(Clone, Debug, Default, PartialEq,
         ::serde::Serialize, ::serde::Deserialize)]
#[serde(rename_all = "camelCase")]
pub struct ListPostsResponse {
    /// Posts returned for the request.
    #[serde(default)]
    pub posts: ::std::vec::Vec<Post>,
}
/// Request for `ListPostComments` RPC method.
#[derive(Clone, Debug, Default, PartialEq,
         ::serde::Serialize, ::serde::Deserialize)]
#[serde(rename_all = "camelCase")]
pub struct ListPostCommentsRequest {
    /// ID of the post parent of the comments to return.
    #[serde(default)]
    pub post: i64,
    /// Returns comments after a given ID.
    #[serde(default)]
    pub after: ::std::option::Option<i64>,
    /// Limit the number of returned posts.
    #[serde(default)]
    pub limit: ::std::option::Option<u32>,
}
/// Response for `ListPostComments` RPC method.
#[derive(Clone, Debug, Default, PartialEq,
         ::serde::Serialize, ::serde::Deserialize)]
#[serde(rename_all = "camelCase")]
pub struct ListPostCommentsResponse {
    /// Comments returned for the request.
    #[serde(default)]
    pub comments: ::std::vec::Vec<self::post::Comment>,
}
/// Request for `CreatePost` RPC method.
#[derive(Clone, Debug, Default, PartialEq,
         ::serde::Serialize, ::serde::Deserialize)]
#[serde(rename_all = "camelCase")]
pub struct CreatePostRequest {
    /// Post to be created
    #[serde(default)]
    pub post: Post,
    /// Author of the post
    #[serde(default)]
    pub author: User,
}
/// Request for `PostComment` RPC method.
#[derive(Clone, Debug, Default, PartialEq,
         ::serde::Serialize, ::serde::Deserialize)]
#[serde(rename_all = "camelCase")]
pub struct PostCommentRequest {
    /// Comment to be posted.
    #[serde(default)]
    pub comment: self::post::Comment,
    /// Author of the comment.
    #[serde(default)]
    pub author: User,
}
/// Request for `LikePost` RPC method.
#[derive(Clone, Debug, Default, PartialEq,
         ::serde::Serialize, ::serde::Deserialize)]
#[serde(rename_all = "camelCase")]
pub struct LikePostRequest {
    /// Post to be liked or disliked.
    #[serde(default)]
    pub post: i64,
    /// If set, dislike the post.
    #[serde(default)]
    pub dislike: bool,
}
/// A user in the service.
#[derive(Clone, Debug, Default, PartialEq,
         ::serde::Serialize, ::serde::Deserialize)]
#[serde(rename_all = "camelCase")]
#[derive(
    :: diesel :: Queryable,
    :: diesel :: QueryableByName,
    :: diesel :: Insertable,
    :: diesel :: Identifiable,
)]
#[table_name = "user_schema"]
#[primary_key(user_id)]
pub struct User {
    /// ID of the user.
    #[serde(default)]
    #[serde(skip_deserializing)]
    #[column_name = "user_id"]
    pub id: i64,
    /// Name of the user.
    #[serde(default)]
    #[column_name = "name"]
    pub name: ::std::string::String,
    /// Idenfier of the user.
    #[serde(default)]
    #[serde(flatten)]
    #[diesel(embed)]
    pub identifier: self::user::Identifier,
}
impl User {
    /// Return the columns in the order specified in the API spec.
    ///
    /// This is useful when selecting in the table and loading to the generated
    /// struct, as diesel
    /// requires that the order of columns match exactly as in the
    /// [`diesel.Queryable`] implementation.
    ///
    /// As the schema defined order may be different than the spec order, this
    /// function solves the
    /// issue by providing always the correct order.
    ///
    /// See [`Self::all`] to a helper that already does the selection for you.
    pub fn columns() -> UserColumns {
        std::default::Default::default()
    }
    /// A query that does the correct selection when deserializing to [`Self`].
    pub fn all() -> ::diesel::helper_types::Select<self::user_schema::table,
                                                   UserColumns>
    {
        self::user_schema::table.select(Self::columns())
    }
}
/// A post in the service.
#[derive(Clone, Debug, Default, PartialEq,
         ::serde::Serialize, ::serde::Deserialize)]
#[serde(rename_all = "camelCase")]
#[derive(
    :: diesel :: Queryable,
    :: diesel :: QueryableByName,
    :: diesel :: Insertable,
    :: diesel :: Identifiable,
)]
#[table_name = "post_schema"]
#[primary_key(post_id)]
#[derive(:: diesel :: Associations)]
# [belongs_to (foreign_key = "author_id" , parent = User)]
#[derive(:: diesel :: AsChangeset)]
pub struct Post {
    /// ID of the post.
    #[serde(default)]
    #[serde(skip_deserializing)]
    #[column_name = "post_id"]
    pub id: i64,
    /// Title of the post.
    #[serde(default)]
    #[column_name = "title"]
    pub title: ::std::string::String,
    /// Likes of the post.
    #[serde(default)]
    #[serde(skip_deserializing)]
    #[column_name = "likes"]
    pub likes: i32,
    /// Dislikes of the post.
    #[serde(default)]
    #[serde(skip_deserializing)]
    #[column_name = "dislikes"]
    pub dislikes: i32,
    /// Author of the post.
    #[serde(default)]
    #[column_name = "author_id"]
    pub author_id: i64,
    /// Content of the post.
    #[serde(default)]
    #[column_name = "content"]
    pub content: ::std::string::String,
    /// Format used in `content`.
    #[serde(default)]
    #[column_name = "content_format"]
    pub content_format: self::post::ContentFormat,
}
impl Post {
    ///  Query all the rows that belongs to a given User FK value.
    pub fn all_of_user(
        author_id: i64,
    ) -> ::diesel::dsl::Filter<
        ::diesel::dsl::Select<self::post_schema::table, PostColumns>,
        ::diesel::dsl::Eq<self::post_schema::author_id, i64>,
    > {
        Self::all().filter(self::post_schema::author_id.eq(author_id))
    }
}
impl Post {
    /// Return the columns in the order specified in the API spec.
    ///
    /// This is useful when selecting in the table and loading to the generated
    /// struct, as diesel
    /// requires that the order of columns match exactly as in the
    /// [`diesel.Queryable`] implementation.
    ///
    /// As the schema defined order may be different than the spec order, this
    /// function solves the
    /// issue by providing always the correct order.
    ///
    /// See [`Self::all`] to a helper that already does the selection for you.
    pub fn columns() -> PostColumns {
        std::default::Default::default()
    }
    /// A query that does the correct selection when deserializing to [`Self`].
    pub fn all() -> ::diesel::helper_types::Select<self::post_schema::table,
                                                   PostColumns>
    {
        self::post_schema::table.select(Self::columns())
    }
}
pub mod user {
    use ::diesel::prelude::*;
    #[doc(hidden)]
    pub type __IdentifierValues = (
        ::std::option::Option<
            ::diesel::helper_types::Eq<
                super::user_schema::email,
                ::std::string::String
            >,
        >,
        ::std::option::Option<
            ::diesel::helper_types::Eq<
                super::user_schema::username,
                ::std::string::String
            >,
        >,
    );
    #[doc(hidden)]
    pub type __IdentifierByRefValues<'insert> = (
        ::std::option::Option<
            ::diesel::helper_types::Eq<
                super::user_schema::email,
                &'insert ::std::string::String
            >,
        >,
        ::std::option::Option<
            ::diesel::helper_types::Eq<
                super::user_schema::username,
                &'insert ::std::string::String,
            >,
        >,
    );
    #[doc(hidden)]
    pub type __IdentifierQueryableRow = (
        ::std::option::Option<::std::string::String>,
        ::std::option::Option<::std::string::String>,
    );
    /// Idenfier of the user.
    #[derive(Debug, Clone, PartialEq,
         ::serde::Serialize, ::serde::Deserialize)]
    #[serde(rename_all = "camelCase")]
    pub enum Identifier {
        /// Default value for the OneOf. Used when no field is set.
        NotSet,
        /// The user used an email to identify himself.
        Email(::std::string::String),
        /// The user used a username to identify himself.
        Username(::std::string::String),
    }
    impl Identifier {
        pub fn is_set(&self) -> bool {
            self != &Self::NotSet
        }
    }
    impl ::std::default::Default for Identifier {
        fn default() -> Self {
            Self::NotSet
        }
    }
    impl ::diesel::Insertable<super::user_schema::table> for Identifier {
        type Values =
            <__IdentifierValues as
                ::diesel::Insertable<super::user_schema::table>
            >::Values;
        fn values(self) -> Self::Values {
            let mut values: __IdentifierValues =
                ::std::default::Default::default();
            match self {
                Self::Email(x) => {
                    values.0 = ::std::option::Option::Some(
                        super::user_schema::email.eq(x)
                    );
                }
                Self::Username(x) => {
                    values.1 = ::std::option::Option::Some(
                        super::user_schema::username.eq(x)
                    );
                }
                _ => {}
            }
            values.values()
        }
    }
    impl<__ST, __DB> ::diesel::Queryable<__ST, __DB> for Identifier
    where
        __DB: ::diesel::backend::Backend,
        __IdentifierQueryableRow: ::diesel::Queryable<__ST, __DB>,
    {
        type Row =
            <__IdentifierQueryableRow as ::diesel::Queryable<__ST, __DB>>::Row;
        fn build(row: Self::Row) -> Self {
            match __IdentifierQueryableRow::build(row) {
                (
                    std::option::Option::Some(x),
                    ::std::option::Option::None,
                ) => Self::Email(x),
                (
                    ::std::option::Option::None,
                    std::option::Option::Some(x),
                ) => Self::Username(x),
                _ => ::std::default::Default::default(),
            }
        }
    }
    impl<'insert> ::diesel::Insertable<super::user_schema::table> for
        &'insert Identifier
    {
        type Values = <__IdentifierByRefValues<'insert> as ::diesel::Insertable<
            super::user_schema::table,
        >>::Values;
        fn values(self) -> Self::Values {
            let mut values: __IdentifierByRefValues<'insert> =
                ::std::default::Default::default();
            match self {
                Identifier::Email(x) => {
                    values.0 = ::std::option::Option::Some(
                        super::user_schema::email.eq(x)
                    );
                }
                Identifier::Username(x) => {
                    values.1 = ::std::option::Option::Some(
                        super::user_schema::username.eq(x)
                    );
                }
                _ => {}
            }
            values.values()
        }
    }
}
pub mod blob_service {
    use ::actix_web::FromRequest as __FromRequest;
    /// List the users in the system.
    pub async fn list_users(
        service: ::actix_web::web::Data<dyn super::BlobService>,
        payload: ::actix_web::web::Query<super::ListUsersRequest>,
        http_req: ::actix_web::HttpRequest,
    ) -> ::std::result::Result<
        ::actix_web::web::Json<super::ListUsersResponse>,
        ::actix_web::Error,
    > {
        let request = payload.into_inner();
        let response = service.list_users(request, http_req).await?;
        ::std::result::Result::Ok(::actix_web::web::Json(response))
    }
    /// List a number of posts in the system.
    pub async fn list_posts(
        service: ::actix_web::web::Data<dyn super::BlobService>,
        payload: ::actix_web::web::Query<super::ListPostsRequest>,
        http_req: ::actix_web::HttpRequest,
    ) -> ::std::result::Result<
        ::actix_web::web::Json<super::ListPostsResponse>,
        ::actix_web::Error,
    > {
        let request = payload.into_inner();
        let response = service.list_posts(request, http_req).await?;
        ::std::result::Result::Ok(::actix_web::web::Json(response))
    }
    /// List comments of a given post.
    pub async fn list_post_comments(
        service: ::actix_web::web::Data<dyn super::BlobService>,
        payload: ::actix_web::web::Query<super::ListPostCommentsRequest>,
        http_req: ::actix_web::HttpRequest,
    ) -> ::std::result::Result<
        ::actix_web::web::Json<super::ListPostCommentsResponse>,
        ::actix_web::Error,
    > {
        let mut request = payload.into_inner();
        let ::actix_web::web::Path((post,)) =
            ::actix_web::web::Path::extract(&http_req).await?;
        request.post = post;
        let response = service.list_post_comments(request, http_req).await?;
        ::std::result::Result::Ok(::actix_web::web::Json(response))
    }
    /// Create a post in the system.
    pub async fn create_post(
        service: ::actix_web::web::Data<dyn super::BlobService>,
        payload: ::actix_web::web::Json<super::CreatePostRequest>,
        http_req: ::actix_web::HttpRequest,
    ) -> ::std::result::Result<
        ::actix_web::web::Json<super::Post>,
        ::actix_web::Error,
    > {
        let request = payload.into_inner();
        let response = service.create_post(request, http_req).await?;
        ::std::result::Result::Ok(::actix_web::web::Json(response))
    }
    /// Post a comment.
    pub async fn post_comment(
        service: ::actix_web::web::Data<dyn super::BlobService>,
        payload: ::actix_web::web::Json<super::PostCommentRequest>,
        http_req: ::actix_web::HttpRequest,
    ) -> ::std::result::Result<
        ::actix_web::web::Json<super::post::Comment>,
        ::actix_web::Error,
    > {
        let mut request = payload.into_inner();
        let ::actix_web::web::Path((post_id,)) =
            ::actix_web::web::Path::extract(&http_req).await?;
        request.comment.post_id = post_id;
        let response = service.post_comment(request, http_req).await?;
        ::std::result::Result::Ok(::actix_web::web::Json(response))
    }
    /// {Dis}Like a post.
    pub async fn like_post(
        service: ::actix_web::web::Data<dyn super::BlobService>,
        payload: ::actix_web::web::Json<super::LikePostRequest>,
        http_req: ::actix_web::HttpRequest,
    ) -> ::std::result::Result<
        ::actix_web::web::Json<super::Post>,
        ::actix_web::Error,
    > {
        let mut request = payload.into_inner();
        let ::actix_web::web::Path((post,)) =
            ::actix_web::web::Path::extract(&http_req).await?;
        request.post = post;
        let response = service.like_post(request, http_req).await?;
        ::std::result::Result::Ok(::actix_web::web::Json(response))
    }
}
pub mod post {
    use std::io::Write as __Write;

    use diesel::prelude::*;

    use crate::schema::comments as comment_schema;
    pub type CommentColumns = (
        self::comment_schema::comment_id,
        self::comment_schema::post_id,
        self::comment_schema::author_id,
        self::comment_schema::content,
    );
    /// A comment on a post in the service.
    #[derive(Clone, Debug, Default, PartialEq,
             ::serde::Serialize, ::serde::Deserialize)]
    #[serde(rename_all = "camelCase")]
    #[derive(
        ::diesel::Queryable,
        ::diesel::QueryableByName,
        ::diesel::Insertable,
        ::diesel::Identifiable,
    )]
    #[table_name = "comment_schema"]
    #[primary_key(comment_id)]
    #[derive(::diesel::Associations)]
    #[belongs_to(foreign_key = "author_id" , parent = super::User)]
    #[belongs_to(parent = super::Post , foreign_key = "post_id")]
    #[derive(::diesel::AsChangeset)]
    pub struct Comment {
        /// ID of the comment.
        #[serde(default)]
        #[serde(skip_deserializing)]
        #[column_name = "comment_id"]
        pub id: i64,
        /// ID of the post where this comment was posted.
        #[serde(default)]
        #[column_name = "post_id"]
        pub post_id: i64,
        /// Author of the comment.
        #[serde(default)]
        #[column_name = "author_id"]
        pub author_id: i64,
        /// Content of the comment.
        #[serde(default)]
        #[column_name = "content"]
        pub content: ::std::string::String,
    }
    impl Comment {
        ///  Query all the rows that belongs to a given User FK value.
        pub fn all_of_user(
            author_id: i64,
        ) -> ::diesel::dsl::Filter<
            ::diesel::dsl::Select<self::comment_schema::table, CommentColumns>,
            ::diesel::dsl::Eq<self::comment_schema::author_id, i64>,
        > {
            Self::all().filter(self::comment_schema::author_id.eq(author_id))
        }
        ///  Query all the rows that belongs to a given Post FK value.
        pub fn all_of_post(
            post_id: i64,
        ) -> ::diesel::dsl::Filter<
            ::diesel::dsl::Select<self::comment_schema::table, CommentColumns>,
            ::diesel::dsl::Eq<self::comment_schema::post_id, i64>,
        > {
            Self::all().filter(self::comment_schema::post_id.eq(post_id))
        }
    }
    impl Comment {
        /// Return the columns in the order specified in the API spec.
        ///
        /// This is useful when selecting in the table and loading to the
        /// generated struct, as diesel
        /// requires that the order of columns match exactly as in the
        /// [`diesel.Queryable`] implementation.
        ///
        /// As the schema defined order may be different than the spec order,
        /// this function solves the
        /// issue by providing always the correct order.
        ///
        /// See [`Self::all`] to a helper that already does the selection for
        /// you.
        pub fn columns() -> CommentColumns {
            std::default::Default::default()
        }
        /// A query that does the correct selection when deserializing to
        /// [`Self`].
        pub fn all() -> ::diesel::helper_types::Select<self::comment_schema::table,
                                                       CommentColumns>
        {
            self::comment_schema::table.select(Self::columns())
        }
    }
    /// Possible formats of content in a Post.
    #[derive(
        Debug,
        Copy,
        Clone,
        PartialEq,
        Eq,
        Hash,
        PartialOrd,
        Ord,
        :: serde :: Serialize,
        :: serde :: Deserialize,
    )]
    #[serde(rename_all = "SCREAMING_SNAKE_CASE")]
    #[derive(:: diesel :: AsExpression, :: diesel :: FromSqlRow)]
    #[sql_type = "ContentFormatSql"]
    pub enum ContentFormat {
        /// Conventional default.
        Unknown = 0,
        /// The post was written in plain text.
        PlainText = 1,
        /// The post was written in markdown.
        Markdown = 2,
    }
    impl ::std::default::Default for ContentFormat {
        fn default() -> Self {
            Self::Unknown
        }
    }
    impl ::diesel::deserialize::FromSql<ContentFormatSql, ::diesel::pg::Pg> for
        ContentFormat
    {
        fn from_sql(
            raw: ::std::option::Option<&[u8]>,
        ) -> ::diesel::deserialize::Result<Self>
        {
            match ::diesel::not_none!(raw) {
                b"UNKNOWN" => ::std::result::Result::Ok(Self::Unknown),
                b"PLAIN_TEXT" => ::std::result::Result::Ok(Self::PlainText),
                b"MARKDOWN" => ::std::result::Result::Ok(Self::Markdown),
                v => ::std::result::Result::Err(
                    format!(
                        "Unrecognized enum variant: '{}'",
                        ::std::string::String::from_utf8_lossy(v)
                    )
                    .into(),
                ),
            }
        }
    }
    impl<__DB> ::diesel::serialize::ToSql<ContentFormatSql, __DB> for
        ContentFormat
    where
        __DB: ::diesel::backend::Backend,
    {
        fn to_sql<W>(
            &self,
            out: &mut ::diesel::serialize::Output<W, __DB>,
        ) -> ::diesel::serialize::Result
        where
            W: ::std::io::Write,
        {
            match *self {
                Self::Unknown => out.write_all(b"UNKNOWN")?,
                Self::PlainText => out.write_all(b"PLAIN_TEXT")?,
                Self::Markdown => out.write_all(b"MARKDOWN")?,
            }
            ::std::result::Result::Ok(::diesel::serialize::IsNull::No)
        }
    }
    #[derive(:: diesel :: SqlType, :: diesel :: QueryId, Clone, Copy)]
    #[postgres(type_name = "content_format", type_schema = "blog")]
    pub struct ContentFormatSql {}
}
\end{minted}
