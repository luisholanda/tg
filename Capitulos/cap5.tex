Neste trabalho, foram apresentados os principais problemas que geradores de código
possuem atualmente. Dentre eles, uma limitação de sua extensibilidade causada
pela própria arquitetura do gerador.

Em seguida, apresentamos a arquitetura de novo gerador de código, \Baker{}, focada
na extensibilidade, e mostramos como essa pode ser utilizada para integrar de forma
arbitrária diversas bibliotecas ou funcionalidades sem a necessidade de construir um
gerador para cada combinação, apenas para cada integração.

Também construímos um experimento que mostra que a arquitetura apresentada
resolve os problemas apresentados nas soluções anteriores, implementando
integrações com \textit{frameworks} Web e ORMs.

O experimento também foi capaz de validar como a existência de um gerador na
caixa de ferramentas de um time é capaz de acelerar o desenvolvimento. $~65\%$
do código total do experimento é gerado a partir da especificação, permitindo
que os desenvolvedores se preocupem apenas com a implementação das regras de
negócio do serviço.

Uma análise da \textit{codebase} do time também mostrou diversas possibilidades
para outros geradores que aumentariam ainda mais a eficiência do time e a
confiabilidade do serviço.

Contudo, ao longo do projeto, diversas simplificações foram feitas e oportunidades
de melhorias foram descobertas. Acreditamos que trabalhos futuros sejam capazes
de utilizar tais oportunidades para melhorar a arquitetura ou a implementação.

\begin{itemize}
\item A implementação utiliza um \textit{parser} próprio para a IDL utilizada,
  trabalhos futuros podem utilizar o compilador original dela para simplificar a
  implementação, melhorar mensagens de erros e o suporte a linguagem. Além disso,
  o uso de estruturas padrões do compilador permitiria que uma melhor integração
  com o resto do ecossistema.
\item Como observado no \cref{cap:project}, o algoritmo para a combinação dos
  diversos fragmentos de IR é simples, não suportando diversos casos de uso
  relevantes. Trabalhos futuros poderiam estudar e formalizar algoritmos mais precisos
  que suportem um número maior de casos.
\item No momento, a IR suporta apenas o que foi necessário para o experimento,
  estudos e melhorias futuras são necessários para suportar um maior número de
  geradores.
\end{itemize}
