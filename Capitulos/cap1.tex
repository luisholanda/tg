\section{Contexto}

O desenvolvimento de aplicações Web apresenta um variado nível de dificuldade e
complexidade: pode variar desde um simples serviço com poucas operações até
um complexo sistema com dezenas (ou centenas) de serviços interligados de maneira
arbitrária com centenas ou milhares de operações.

Nesse contexto, diversas ferramentas foram desenvolvidas para tentar facilitar
e simplificar o desenvolvimento desses sistemas: geradores de APIs e \textit{object-relational
mapping} (ORMs) são exemplos de ferramentas que caem nessa categoria.

\section{\textit{Interface Definition Languages}}

APIs são normalmente especificadas em \textit{Interface Definition Languages}
(IDLs). Essas linguagens possuem construções que facilitam o desenvolvimento de uma
especificação, em relação a operações, entradas, saídas e erros. Qual é usada
depende do tipo de API que está sendo construída, alguns exemplos são:

\begin{itemize}
\item
  \textit{Web Service Description Language} \cite{wsdl:spec}: usada para
    definir APIs que usam o protocolo SOAP. É uma linguagem baseada em XML.
    Um exemplo de um arquivo WSDL é apresentado no \cref{anex:wsdl-example}.
\item
  \textit{OpenAPI} \cite{openapi:spec}: comumente usada para definir APIs que usam
    o protocolo HTTP \cite{rfc2616}, informalmente chamadas de \textit{"Restful APIs"}
    em referência ao conceito de REST definido em \cite{10.5555/932295}. É uma linguagem
    baseada em YAML. Um exemplo de um arquivo OpenAPI é apresentado no \cref{anex:openapi-example}.
\item
  \textit{Protocol Buffers} \cite{googl:protobuf}: usada para definir APIs que usam
    o protocolo gRPC \cite{googl:grpc}. O nome se refere tanto à IDL usada para a
    especificação quanto para ao formato binário usado para transmitir as mensagens.
    Um exemplo de um arquivo em Protocol Buffers é apresentado no \cref{anex:protobuf-example}.
\end{itemize}

Existem diversas vantagens em usar uma IDL, entre elas:

\begin{enumerate}
\item
  A comunicação entre diferentes times (possivelmente em diferente organizações) que
  precisam interagir via API é mais simples, já que todos os detalhes de interface
  estão especificados em um formato padrão.
\item
  Por esse motivo, é muito mais simples construir ferramentas de análise sobre a
  especificação, como geradores de código, validadores, ferramentas de teste, etc.
\end{enumerate}

\section{Geradores de APIs}

Geradores de APIs são ferramentas que, via uma especificação de uma API, conseguem
produzir código-fonte em uma linguagem de programação tanto para realizar requisições
a essa API, quanto para gerar uma base para a implementação da API em si.
O código gerado em ambos os casos contém, tanto métodos para as operações suportadas
pela API, quanto as estruturas de dados necessárias para interfacear com as mensagens
a serem transmitidas.

Existem diversas vantagens em usar um gerador de API, entre elas:

\begin{enumerate}
\item
  Como toda a parte dos modelos são geradas, tanto pelo servidor, como pelo cliente,
  há uma significativa redução no volume de linhas de código-fonte do sistema, reduzindo
  a possibilidade de defeitos \cite{5010260} e facilitando o entendimento do projeto por
  novos desenvolvedores.
\item
  Como o código é gerado a partir da especificação, sabemos que a implementação
  vai estar de acordo com a interface especificada, permitindo que o programador
  foque em implementar a lógica de cada operação, otimizando o uso do tempo. No
  caso de consumidores, eles poupam o trabalho de ter que implementar um código
  de integração com a API, que pode conter erros e ser difícil de manter,
  principalmente com relação a mudanças e adições na API.
\item
  O código gerado abstrai toda a camada de comunicação e rede, tanto do servidor
  quanto do cliente, facilitando o entendimento do código que o programador precisa
  implementar.
\end{enumerate}

Existem diversos exemplos de geradores de APIs, alguns são \cite{openapi:gen} e
\cite{googl:protobuf}.

\section{ORMs}

Da mesma forma que IDLs e geradores tentam facilitar o desenvolvimento da
\textit{interface} de uma API, ORMs tentam facilitar a integração do código da
API com o banco de dados usado (seja ele SQL ou NoSQL). Elas são bibliotecas que
abstraem a execução de operações no banco de dados em uma interface amigável para
a linguagem de programação usada. Por esse motivo, ORM é específica para a linguagem
de programação em que ela foi implementada, e normalmente também é específica para
um banco ou conjunto de bancos.

ORMs normalmente funcionam via anotações e interfaces que o programador precisa
adicionar ou implementar no código-fonte dos modelos. Essas anotações servem para,
por exemplo, mapear o modelo a uma tabela, um campo a uma coluna, ou uma relação
com outro modelo (1-1, 1-muitos ou muitos-muitos).

Um exemplo em Python usando a ORM \texttt{sqlalchemy} é \cref{lst:sqlalchemy-example}.

\begin{listing}[ht]
\begin{minted}{python}
from sqlalchemy import Column, DateTime, String, Integer, ForeignKey, func
from sqlalchemy.orm import relationship, backref
from sqlalchemy.ext.declarative import declarative_base


Base = declarative_base()


class Department(Base):
    __tablename__ = 'department'
    id = Column(Integer, primary_key=True)
    name = Column(String)


class Employee(Base):
    __tablename__ = 'employee'
    id = Column(Integer, primary_key=True)
    name = Column(String)
    # Use default=func.now() to set the default hiring time
    # of an Employee to be the current time when an
    # Employee record was created
    hired_on = Column(DateTime, default=func.now())
    department_id = Column(Integer, ForeignKey('department.id'))
    # Use cascade='delete,all' to propagate the deletion of a Department
    # onto its Employees
    department = relationship(
        Department,
        backref=backref('employees',
                         uselist=True,
                         cascade='delete,all'))

\end{minted}
\caption{Exemplo de código usando \texttt{sqlalchemy}}
\label{lst:sqlalchemy-example}
\end{listing}

ORMs são populares pois simplificam o trabalho do desenvolvedor ao automatizar muitas
operações que são comumente realizadas no banco, e por prover uma DSL caso seja
necessário fazer algo mais complexo. Dependendo da linguagem, essas funcionalidades
podem ser validadas em tempo de compilação, previnindo defeitos.

\section{Problema}

Apesar de serem ferramentas muito populares, geradores de APIs e ORMs não integram
bem. O primeiro costuma focar na \textit{interface de comunicação}, não prestando
atenção a detalhes de implementação. Além disso, dado o grande número de ORMs
presente para cada linguagem, e também as diversas formas de se gerar a interface
de comunicação, geradores convencionais não conseguiriam adicionar suporte para
todas as combinações possíveis.

Outro problema em como os geradores são implementados hoje é que eles possuem
suporte limitado a extensões externas ao código-fonte. Dependendo do gerador
utilizado, é necessário implementar um \textit{novo gerador}, o que gera um grande
custo operacional. Alguns exemplos de modificações possíveis:

\begin{itemize}
\item
  Geração automática de testes para as operações \cite{9159071}.
\item
  Validação automática de propriedades das mensagens \cite{envoy:protoc-gen-validate}.
\item
  Integração com ORMs ou outras bibliotecas.
\end{itemize}

Devido a esses problemas, muitas organizações deixam de usar essas ferramentas e os
programadores precisam implementar manualmente códigos que poderiam ser gerados.
Isso aumenta a chance de erros ocorrerem durante a implementação, e o resultado divergir
da especificação. Além disso, diminui a eficiência do time, pois há mais tarefas a
serem realizadas.

Avaliando as tarefas realizadas pelo time de engenharia de uma organização durante o
ano de 2020, foi possível identificar que pelos menos 30\% das tarefas realizadas
eram relacionadas a implementação de modelos, integração com ORMs e com a camada
de comunicação da API. Além disso, dentro desses 30\%, ocorreram diversas vezes tarefas
extras relacionadas com ajustes da implementação para que essa seguisse a especificação.

Na tentativa de solucionar esse problema, esse trabalho propõe um novo modelo de
gerador de APIs, que pode ser extendido para suportar qualquer linguagem ou funcionalidade
nova sem a necessidade de modificar o código-fonte.

Esse trabalho é estruturado como se segue. \cref{cap:past-works} faz uma análise de
trabalhos anteriores. \cref{cap:project} apresenta, de forma detalhada, o que foi
construído e o experimento realizado. \cref{cap:validation} apresenta como a implementação
e o experimento foram validados. Por fim, concluímos no \cref{cap:conclusion}.
