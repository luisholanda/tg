\section{Experimento}

Para experimentarmos a plataforma, identificar pontos de melhoria durante a construção e
servir como base para o experimento de validação, construimos uma série de \textit{layers} e
um \textit{codegen} para a linguagem de programação Rust \cite{rust:site}. Os geradores foram
escolhidos com o objetivo de permitir a usuários construir serviços Web utilizando a plataforma.

\subsection{\textit{Layers} Definidas}

Mantendo a ideia de componentização dos geradores, cada \textit{layer} é responsável por apenas
parte do código gerado: cada uma gera a integração com uma biblioteca específica. Como a troca
de uma \textit{layer} por outra é simples, é possível tomar decisões por padrão sem limitar o
usuário.

\subsubsection{\texttt{rust-types}}

A primeira \textit{layer} desenvolvida, e a primeira que deve ser passada para o programa, é
o gerador responsável por traduzir as mensagems definidas nos arquivos fonte em tipos da linguagem
objeto. Esse deve ser o primeiro programa a ser passado devido as regras de combinação de IR: o
primeiro resultado é aquele que tem prioridade na definição de campos dos tipos.

Um exemplo simples de código gerado é apresentado pelas listagens \ref{lst-rust-types-proto-src}
e \ref{lst-rust-types-rust-out}, nas quais uma \textit{struct} em Rust é gerada a partir de uma
\textit{message} em Protocol Buffers. Um ponto a ser comentado é que, apesar de estarmos construindo
um gerador de código, isso não significa que não possamos utilizar mecânicas de geração da própria
linguagem objeto. No nosso exemplo, o uso do atributo \texttt{derive} permite que mais código seja
gerado pelo compilador da linguagem (simplificando o código gerado e facilitando o entendimento
dos desenvolvedores).

\begin{listing}
\begin{minted}{protobuf}
// Response of the [`ListPets`] RPC method.
message ListPetsResponse {
  // A paged array of pets.
  repeated Pet pets = 1;
  // A link to the next page of responses.
  optional string cursor = 2;
}
\end{minted}
\caption{Exemplo de entrada para a \textit{layer} \texttt{rust-types}}
\label{lst-rust-types-proto-src}
\end{listing}

\begin{listing}
\begin{minted}{rust}
/// Response of the [`ListPets`] RPC method.
#[derive(Clone, Debug, Default, PartialEq)]
struct ListPetsResponse {
    /// A paged array of pets
    pub pets: Vec<Pet>,
    /// A link to the next page of responses
    pub cursor: Option<String>
}
\end{minted}
\caption{Exemplo de saida da \textit{layer} \texttt{rust-types}}
\label{lst-rust-types-rust-out}
\end{listing}

Um exemplo mais complexo é apresentado nas listagens \ref{lst-rust-types-cplx-proto-src} e
\ref{lst-rust-types-cplx-rust-out}. Nesse exemplo, mostramos o código gerado para um \texttt{enum}
e uma mensagem com campo \texttt{oneof}. No caso do \texttt{oneof}, é necessário gerar um
\texttt{enum} adicional que garante que apenas um dos campos do \texttt{oneof} está presente,
ou que nenhum esteja (membro \texttt{NotSet}).

\begin{listing}
\begin{minted}{protobuf}
// Possible formats of content in a Post.
enum ContentFormat {
  // Conventional default.
  UNKNOWN = 0;
  // The post was written in plain text.
  PLAIN_TEXT = 1;
  // The post was written in markdown.
  MARKDOWN = 2;
}

// A user in the service.
message User {
  // ID of the user.
  int64 id = 1;
  // Name of the user.
  string name = 2;

  // Idenfier of the user.
  oneof identifier {
    // The user used an email to identify himself.
    string email = 3;
    // The user used a username to identify himself.
    string username = 4;
  };
}
\end{minted}
\caption{Exemplo Complexo de entrada para a \textit{layer} \texttt{rust-types}}
\label{lst-rust-types-cplx-proto-src}
\end{listing}

\begin{listing}
\begin{minted}{rust}
/// Possible formats of content in a Post.
#[derive(Clone, Copy, Debug, PartialEq, PartialOrd, Eq, Ord, Hash)]
enum ContentFormat {
    /// Conventional default.
    Unknown = 0,
    /// The post was written in plain text.
    PlainText = 1,
    /// The post was written in markdown.
    Markdown = 2,
}

impl Default for ContentFormat {
   fn default() -> Self {
      Self::Unknown
   }
}

/// A user in the service.
struct User {
    /// ID of the user.
    pub id: i64,
    /// Name of the user.
    pub name: String,
    /// Identifier of the user.
    pub identifier: user::Identifier
}

pub mod user {
    /// Idenfier of the user.
    pub enum Identifier {
        /// Default value for the OneOf. Used when no field is set.
        NotSet,
        /// The user used an email to identify himself.
        Email(String),
        /// The user used a username to identify himself.
        Username(String)
    }

    impl Identifier {
        pub fn is_set(&self) -> bool {
            self != &Self::NotSet
        }
    }

    impl Default for Identifier {
        fn default() -> Self {
            Self::NotSet
        }
    }
}
\end{minted}
\caption{Exemplo Complexo de saida da \textit{layer} \texttt{rust-types}}
\label{lst-rust-types-cplx-rust-out}
\end{listing}

\subsubsection{\texttt{rust-serde}}

Outra \textit{layer} definida é aquela que integra o código gerado com a biblioteca
de serialização/deserialização \texttt{serde} \cite{rust:serde}, a biblioteca padrão
para essa funcionalidade. A integração é feita via diversos atributos adicionados a
tipos, campos e membros. Por padrão, os campos são recebidos e enviados em \texttt{camelCase}
e valores de \texttt{enum}s em \texttt{SCREAMING\_SNAKE\_CASE}, mas o \textit{casing}
pode ser escolhido utilizado as opções \texttt{baker.api.*\_name\_case}. Os nomes também
podem ser renomeados com as opções \texttt{baker.api.*\_name}.

\subsubsection{\texttt{rust-diesel}}

Para demonstrar que é possível integrar uma ORM ao código gerado sem dificuldades, uma
\textit{layer} foi construída para integrar a ORM \texttt{diesel} \cite{rust:diesel},
uma ORM para bancos relacionais SQL. Para a integração, atributos, \texttt{ImplBlock}s
e diversas opções são necessários, como pode ser visto nos \cref{anex:blog-spec} e
\cref{anex:blog-codegen}.

O código gerado não somente integra com a biblioteca, mas também fornece uma série de
utilidades para o programador que não são disponibilizadas normalmente, facilitando o
uso do código gerado. Entre essas utilidades, está a criação de um tipo que pode ser
utilizado nos \textit{schemas} das tabelas em colunas que armazenam \texttt{enum}s
(\texttt{ContentFormatSql} no código gerado do \cref{anex:blog-codegen}).

\subsubsection{\texttt{rust-actix}}

Por fim, a última \textit{layer} desenvolvida integra com o \textit{framework} Web
\texttt{actix-web} \cite{rust:actix-web}. A integração se preocupa em mapear
\texttt{service}s definidos nos arquivos fonte em uma interface que o programador
implemente sem se preocupar com detalhes do \textit{framework}. Dessa forma, todos
os detalhes do contrato da API são gerados automaticamente e o desenvolvedor precisa
somente se preocupar com a lógica de cada operação.

Para configurar a integração, opções nos métodos \texttt{rpc} do serviço são adicionadas,
descrevendo a rota de cada operação (junto com quaisquer variáveis de rota) e o método
HTTP necessário. Para cada operação então, é gerado uma função que recebe a requisição
HTTP, monta o objeto parâmetro a operação, chama o método correto da interface e converte
a resposta do método em uma resposta HTTP.

Para montar o serviço no servidor, é provido um método \texttt{configure} na interface,
que é responsável por criar todas as rotas no servidor (\texttt{trait BlobService} no
código gerado do \cref{anex:blog-codegen}).

\subsection{\texttt{rust-codegen}}

O último programa desenvolvido é o \textit{codegen} para a linguagem. Como podemos
implementar o programa na própria linguagem, é possível utilizar tais ferramentas de
análise (e.g. \textit{parser} e ASTs). Com essas ferramentas em mão, implementar o programa
é uma questão de mapear a IR para a AST da linguagem. Por fim, para facilitar a leitura
do arquivo final, o programa também executa um formatador de código no mesmo.
