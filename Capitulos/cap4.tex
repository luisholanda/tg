Para validarmos a viabilidade da solução, o funcionamento dos geradores implementados e
a qualidade do código resultante, um serviço Web simples foi implementado. Esse serviço
implementa um pequeno blog público, em que usuários podem postar mensagens, comentários
e votar em outras postagens. Mantendo a API simples, conseguimos focar melhor no impacto
do código gerado. A especificação da API é apresentada no \cref{anex:blog-spec}.

Em seguida, foi realizada uma pesquisa com alguns desenvolvedores da empresa TerraMagna,
para que esses julguassem a qualidade do código gerado. Por fim, foi feita uma análise de
outras oportunidades de geradores que facilitariam a implementação de serviços no time.

\section{Implementação do Serviço Web}

Na implementação do serviço, o código gerado é responsável por 64.8\% das linhas de código
total da implementação (633 de 976 linhas). O código gerado é apresentado na íntegra no
\cref{anex:blog-codegen}. Dessas 633 linhas:

\begin{itemize}
\item 135 são da integração com o \textit{framework} web;
\item 228 são da integração com a ORM;
\item os 270 restantes são definições de estruturas.
\end{itemize}

O código implementado manualmente é responsável apenas pela regra de negócio da aplicação,
dispensando preocupação com detalhes de interface da API ou diversos detalhes de bancos
de dados.

À medida que a complexidade do serviço aumenta e a quantidade de regras de negócio também,
a porcentagem total do código gerado diminui. Contudo, outras vantagens se mantêm:

\begin{itemize}
\item O programador não precisa se preocupar em implementar as estruturas manualmente com
  base na especificação. Em situações multi-linguagem ou multi-repositórios, garantir que
  as implementações manuais obedeçam o contrato especificado torna-se complexo. Esse tipo
  de situação é muito comum, principalmente durante o processo de evolução da API.
\item Pelo mesmo motivo, testes de contrato não são, em grande parte, necessários, pois o
  serviço foi gerado a partir da especificação. Apenas testes de contrato para erros são
  necessários, caso o time os faça, já que a especificação não declara tais casos.
\item A especificação continua sendo a fonte da verdade: em sistemas complexos, uma grande
  quantidade de APIs são implementadas e disponibilizadas, ter a garantia de que as operações
  especificadas são as que estão disponíveis facilita a entrada de novos desenvolvedores no
  time e o descobrimento das operações pelos clientes.
\end{itemize}

\section{Qualidade do Código Gerado}

Quando apresentados a um gerador de código, um dos fatores que dificultam o uso é a qualidade
do código gerado e o quão complexo ele é. Na situação em que o time decide deixar de utilizar
o gerador, os desenvolvedores devem manter o código que foi gerado, quanto mais complexo este
for, mais difícil é esse processo.

Para validarmos que essa preocupação não afetaria o uso dos geradores implementados, uma
pesquisa qualitativa foi realizada com membros do time de desenvolvimento da TerraMagna com
um variado grau de senioridade. Nessa pesquisa, após apresentados os \cref{anex:blog-spec} e
\cref{anex:blog-codegen}, foram feitas as seguintes perguntas:

\begin{enumerate}
\item Alguma parte do código que deixa claro que ele foi gerado?
\item É possível entender o que o código faz e como ele é usado?
\item Na hipótese que o gerador deixe de ser usado, você acharia complexo limpar o código?
\end{enumerate}

As respostas à pesquisa foram bem similares. O fato dos caminhos serem absolutos deixa claro
que o código foi gerado, pois programadores em geral importam itens específicos ou criam
\textit{aliases} para caminhos extensos. Contudo, o código continua sendo simples de se
entender e usar. Todos os entrevistados declararam que conseguiriam limpar o código sem
grandes dificuldades, principalmente contando com a ajuda de um bom editor de texto. Além
disso, todos também comentaram que prefeririam usar o código gerado para não terem que
escrever códigos repetitivos.

Caminhos absolutos são utilizados na implementação atual dos geradores para evitarem conflitos
com estruturas declaradas na especificação. Uma melhoria futura nos geradores pode utilizar a
lista de estruturas para determinar se um caminho absoluto é necessário ou não. Essa foi a
principal observação e sugestão feita pelos entrevistados.

\section{Oportunidades}

Apesar de os geradores atuais já facilitarem a implementação de serviços Web, é necessário
entender se há oportunidades para outros geradores. A falta de oportunidades mostraria que
o uso de geradores em ambientes \textit{enterprise} é, de fato, limitado.

Analisando o código dos serviços internos da empresa, encontramos as seguintes oportunidades:

\begin{itemize}
\item Suportar outros sistemas de bancos de dados que não utilizam \texttt{diesel}, como
  Firestore \cite{google:megastore}.
\item Suportar a declaração de mensagems para o sistema de mensageria interno da empresa.
\item Permitir a especificação de requisitos de autorização para os métodos \texttt{rpc}
  seguindo o padrão da empresa, e.g. a operação \texttt{CreateFoo} requer a permissão
  \texttt{BAR.FOO.WRITE}.
\item Declarar condições de estabilidade das operações: \textit{alpha}, \textit{beta},
  \textit{internal only}, \textit{external}. Essas condições são importantes para testes
  e evolução das APIs sem impactar clientes.
\item Integração com o sistema de monitoramento e observabilidade automático para as operações.
\end{itemize}

Essas oportunidades são apenas aquelas que impactam a implementação do serviço \textit{per se}.
Diversas outras oportunidades podem existir quando também consideramos que podemos utilizar
as especificações para gerar clientes, como integrações com bibliotecas de armazenamento em
memória (e.g. \textit{React Redux}).

Com essa lista, podemos perceber que existem diversas formas que o projeto implementado pode
simplificar a implementação dos serviços, aumentar a segurança e confiabilidade dos serviços
e também diminuir a carga cognitiva dos desenvolvedores envolvidos no sistema.
