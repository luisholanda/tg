% TODO: maybe talk about LLVM here?

Durante o desenvolvimento desse trabalho, analisamos diversos trabalhos já publicados
relacionados a geradores de APIs, e extensões a IDLs.

\cite{openapi:gen} apresenta uma vasta gama de geradores baseados na IDL OpenAPI. Até
a presente data, apresenta 66 geradores de clientes para 33 linguagens e 41 geradores
de servidores para 16 linguagens. Esse é o principal programa utilizado para gerar
código em projetos que usam OpenAPI para especificação de suas APIs.

Analisando o código-fonte e documentação do projeto, chegamos às seguintes conclusões:

\begin{itemize}
\item
  Os geradores funcionam com base em um modelo genérico baseado em OpenAPI. O processo
  de geração ocorre da seguinte maneira:

  \begin{enumerate}
  \item
    O programa lê a especificação OpenAPI e a transforma no modelo genérico.
  \item
    A classe do gerador modifica esse modelo da forma que precisar, possivelmente
    adicionando propriedades específicas para ele.
  \item
    O programa envia o modelo final para um motor de \textit{templates}, que
    carrega os templates do gerador específico e os renderiza. O resultado dessa
    etapa é o código final.
  \end{enumerate}

  Esse fato pode acabar por limitar a expressividade do gerador, pois o modelo
  OpenAPI não é capaz de expressar, de forma simples, todos os detalhes envolvidos
  em uma linguagem de programação.
\item
  Apesar de ser possível criar um novo gerador customizado, por exemplo, para uma
  linguagem que o projeto não suporte, sem a necessidade de modificar o programa
  em si, os geradores são estruturas monolíticas. Não é possível implementar uma
  funcionalidade nova, e.g. um novo processo de validação, que possa ser usado por
  todos os geradores. Isso limita significativamente a extensibilidade do projeto,
  além de aumentar a carga operacional nessas situações.
\end{itemize}

\cite{googl:protobuf} é o compilador de Protocol Buffers. Ele apresenta uma estrutura
bastante interessante em questão de extensibilidade: o sistema de \textit{plugins}.
Um \textit{plugin} é um programa que recebe uma mensagem \texttt{CodeGeneratorRequest}
como entrada e escreve na saída uma mensagem \texttt{CodeGeneratorResponse}. As
definições dessas duas mensagens são apresentadas em \cref{lst:code-gen-req} e
\cref{lst:code-gen-res}, respectivamente.

\begin{listing}[ht]
\caption{Especificação de \texttt{CodeGeneratorRequest}}
\label{lst:code-gen-req}
\begin{minted}{protobuf}
message CodeGeneratorRequest {
  repeated string file_to_generate = 1;

  optional string parameter = 2;

  repeated FileDescriptorProto proto_file = 15;

  optional Version compiler_version = 3;
}
\end{minted}
\end{listing}

\begin{listing}[ht]
\caption{Especificação de \texttt{CodeGeneratorResponse}}
\label{lst:code-gen-res}
\begin{minted}{protobuf}
message CodeGeneratorResponse {
  optional string error = 1;

  message File {
    optional string name = 1;

    optional string insertion_point = 2;

    optional string content = 15;
  }

  repeated File file = 15;
}
\end{minted}
\end{listing}

Esse sistema é interessante por dois fatores:

\begin{itemize}
\item
  É muito simples adicionar suporte a uma nova linguagem, precisamos apenas
  implementar um \textit{plugin}. Ponto em comum com o trabalho anterior.
\item
  Usando o campo \texttt{file.insertion\_point}, é possível injetar conteúdo
  de um gerador em outro. O segundo gerador pode adicionar esses pontos no
  arquivo gerado por ele, permitindo que outros geradores modifiquem o resultado
  final.

  Isso soluciona, em parte, o problema de adicionar novas funcionalidades em
  um gerador presente em \cite{openapi:gen}. Dois problemas ainda persistem:

  \begin{enumerate}
  \item
    Estamos limitados aos pontos de inserção disponibilizados pelo gerador, o
    que pode variar de uma linguagem para outra. O quão grave é esse problema
    depende do que se quer fazer com o \textit{plugin} e qual é a linguagem objeto.
  \item
    O sistema trabalha em termos de texto (campo \texttt{file.content}). Isso
    limita o quão genérico nossa funcionalidade pode ser, e.g. um \textit{plugin} de
    validação poderia ser genérico caso o resultado fosse mais estruturado.

    Um exemplo de plugin que poderia ser genérico é \cite{envoy:protoc-gen-validate},
    que provê uma extensão para diversas validações. Hoje, ela é limitada para
    as linguagens Go, C++ e Java. Caso o resultado fosse mais estruturado, seria
    possível implementar tal funcionalidade de forma genérica para uma grande
    quantidade de linguagens.
  \end{enumerate}
\end{itemize}

\cite{9159071} se propõe a resolver um problema diferente dos trabalhos anteriores:
ele gera testes baseado em \textit{Property Based Testing} \cite{10.1145/351240.351266}
a partir de especificações OpenAPI que validam que as respostas da API seguem as
propriedades e formatos especificados. O programa é capaz de gerar esses testes
sem a necessidade de nenhuma extensão a especificação.

\cite{sferruzza:hal-01868498} propõe extensões e um gerador para OpenAPI que é
capaz de modelar como uma operação é implementada. O trabalho cria o conceito de
\textit{componentes atômicos e compostos}: componentes atômicos recebem parâmetros
e podem gerar novos valores e componentes compostos fazem a composição de diversos
componentes para definir uma dada lógica. O programa então é capaz de validar que
as definições e uso dos componentes são válidas, tanto em questão de todas as
variáveis estarem disponíveis quanto que os tipos estão corretos. O gerador é capaz
de gerar código que define esses componentes.

Por fim, \cite{r2c:semgrep} é uma ferramenta de análise estática que suporta uma
quantidade impressionante de linguagens. O diferencial dessa ferramenta é que
o usuário pode criar novas regras de análise utilizando uma DSL implementada sobre
YAML, permitindo com que a mesma regra seja aplicada em diversas linguagens. Isso
é possível pois todas as linguagens que ela suporta são convertidas para
\textit{uma mesma linguagem intermediária} antes das regras serem aplicadas. Essa
funcionalidade é similar a situação em que queremos implementar uma funcionalidade
no nosso gerador de forma independente da linguagem objeto.
