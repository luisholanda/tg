A hipótese base desse trabalho é que, baseado nos projetos mencionados anteriormente,
é possível implementar um novo modelo de gerador que seja muito mais extensível, amigável
e que consiga implementar funcionalidades mais genéricas. Além disso, outra hipótese
é que, com base nessa extensibilidade, conseguimos aumentar de forma considerável a
eficiência de um time de desenvolvimento, utilizando o gerador para automatizar
trabalhos repetitivos.

O objetivo desse trabalho então é implementar esse novo modelo, além de uma série de
funcionalidades a partir do sistema de extensibilidade. Para validar que atingimos nosso
objetivo, implementaremos uma nova API em nível prova de conceito, utilizando esse gerador,
e verificaremos o que seria necessário a mais para a implementação, caso não tivessemos
acesso a tal ferramenta.

A \cref{tbl:cronograma} apresenta uma estimativa das etapas a serem realizadas durante
o desenvolvimento do trabalho.

\begin{table}[ht]
\centering
\begin{tabular}{c | c}
Passo & Período Estimado \\
\hline
\textit{Parser Protocol Buffers} e Linguagem Intermediária & Maio \\
Análises Semânticas sobre a LI & Junho \\
Implementação da interface de \textit{plugins} & Julho \\
\textit{Plugin} para HTTP & Inicio Agosto \\
\textit{Plugin} para ORM & Agosto/Setembro \\
Gerador para linguagem-objeto escolhida & Final Setembro \\
Implementação da nova API em POC & Inicio Outubro \\
Análise dos resultados & Final de Outubro
\end{tabular}
\caption{Cronograma de atividades}
\label{tbl:cronograma}
\end{table}


Para validar o quanto a ferramenta auxilia no desenvolvimento, iremos desenvolver um
serviço simples de um \textit{micro-blog}, com usuários, postagens e curtidas utilizando
o que foi desenvolvido. A partir dessa implementação, analisaremos a quantidade de
\textit{trabalho extra} que seria necessário, caso não tivessemos acesso a um gerador
de código integrado. Com isso, saberemos o ganho de eficiência dado pela ferramenta.

