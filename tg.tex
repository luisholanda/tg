%%% exemplo de utilização da classe ita
%%%
%%%   por        fábio fagundes silveira   -  ffs [at] ita [dot] br
%%%              benedito c. o. maciel     -  bcmaciel [at] ita [dot] br
%%%              giovani volnei meinertz   -  giovani [at] ita [dot] br
%%%    	         hudson alberto bode       -  bode [at] ita [dot]br
%%%    	         p. i. braga de queiroz    -  pi [at] ita [dot] br
%%%    	         jorge a. b. gripp         -  gripp [at] ita [dot] br
%%%    	         juliano monte-mor         -  jamontemor [at] yahoo [dot] com [dot] br
%%%    	         tarcisio a. b. gripp      -  tarcisio.gripp [at] gmail [dot] com
%%%
%%%   versão para overleaf:
%%%   por           alejandro a. rios cruz - aarc.88@gmail.com
%%%                 saulo gómez            - sagomezs@unal.edu.co
%%%  importante: o texto contido neste exemplo nao significa absolutamente nada.  :-)
%%%              o intuito aqui eh demonstrar os comandos criados na classe e suas
%%%              respectivas utilizacoes.
%%%
%%%  tese.tex  2016-08-25
%%%  $headurl: http://www.apgita.org.br/apgita/teses-e-latex.php $
%%%
%%% italus
%%% instituto tecnológico de aeronáutica --- ita, sao jose dos campos, brasil
%%%                   http://groups.yahoo.com/group/italus/
%%% discussion list: italus {at} yahoogroups.com
%%%
%++++++++++++++++++++++++++++++++++++++++++++++++++++++++++++++++++++++++++++++
% para alterar o tipo de documento, preencher a linha abaixo \documentclass[?]{?}
%   \documentclass[tg]{ita}			= trabalho de graduacao
%   \documentclass[tgfem]{ita}	= para engenheiras
%   								msc     		= dissertacao de mestrado
%   								mscfem   		= para mestras
%   								dsc      		= tese de doutorado
%   								dscfem   		= para doutoras
%   								quali    		= exame de qualificacao
%   								qualifem 		= exame de qualificacao para doutoras
% para 'draft version'/'versao preliminar' com data no rodape, adicionar 'dv':
%   \documentclass[dsc, dv]{ita}
% para trabalhos em inglês, adicionar 'eng':
%   \documentclass[dsc, eng]{ita}
%		\documentclass[dsc, eng, dv]{ita}
%++++++++++++++++++++++++++++++++++++++++++++++++++++++++++++++++++++++++++++++
\documentclass[tg]{ita}    % ita.cls based on standard book.cls
% quando alterar a classe, por exemplo de [msc] para [msc, eng]) rode mais uma vez o botão build output caso haja erro
\usepackage{ae}
\usepackage{graphicx}
\usepackage{epsfig}
\usepackage{amsmath}
\usepackage{amssymb}
\usepackage{subfig}
\usepackage{multirow}
\usepackage{float}
\usepackage{amsthm}
\usepackage{url}         % formats url addresses properly
\usepackage{appendix}    % allows appendix section to be included
\usepackage{lscape}      % allows a page to be rendered in landscape mode
\usepackage{multicol}    % allows text in multi columns
\usepackage{cancel}      % needed to show canceled terms in equations
\usepackage{lettrine}
\usepackage{float}
\usepackage{placeins}
\usepackage[outputdir=latex.out]{minted}
\usepackage{prettyref}

\renewcommand\listingscaption{Código}

\newrefformat{anex}{Anexo~\ref{#1}}
\newrefformat{cap}{Capítulo~\ref{#1}}
\newrefformat{lst}{Código~\ref{#1}}
\newrefformat{tbl}{Tabela~\ref{#1}}

% Make ref autocomplete work.
\newcommand{\cref}[1]{\prettyref{#1}}

\usemintedstyle{friendly}

%HHHHHHHHHHHHHHHHHHHHHHHHHHHHHHHHHHHHHHHHHHHHHHHHHHHHHHHHHHHHHHHHHHHHHHHHHHHHHHHHHHHHHHHHHHHHHHHHHHHHHHHHHHHH
%\usepackage{subfigure}
%\usepackage{subfigmat}
%PACOTEFIGURAS_SE _ERRADO_ESXCLUIR_ACIMA
\usepackage{booktabs}
%PACOTETABELAS_SE _ERRADO_ESXCLUIR_ACIMA
%HHHHHHHHHHHHHHHHHHHHHHHHHHHHHHHHHHHHHHHHHHHHHHHHHHHHHHHHHHHHHHHHHHHHHHHHHHHHHHHHHHHHHHHHHHHHHHHHHHHHHHHHHHHH

%++++++++++++++++++++++++++++++++++++++++++++++++++++++++++++++++++++++++++++++
% Espaçamento padrão de todo o documento
%++++++++++++++++++++++++++++++++++++++++++++++++++++++++++++++++++++++++++++++
\onehalfspacing

%singlespacing Para um espaçamento simples
%onehalfspacing Para um espaçamento de 1,5
%doublespacing Para um espaçamento duplo

%++++++++++++++++++++++++++++++++++++++++++++++++++++++++++++++++++++++++++++++
% Identificacoes (se o trabalho for em inglês, insira os dados em inglês)
% Para entradas abreviadas de Professora (Profa.) em português escreva: Prof$^\textnormal{a}$.
%++++++++++++++++++++++++++++++++++++++++++++++++++++++++++++++++++++++++++++++
\course{Engenharia da Computação}

% Autor do trabalho: Nome Sobrenome
\authorgender{masc}
\author{Luis Cláudio Magalhães de}{Holanda}
\itaauthoraddress{Rua Engenheiro Prudente Merieles de Morais, 813, apto 806}{12.243-750}{São José dos Campos--SP}

% Titulo da Tese/Dissertação
\title{Aplicação de Model Driven Development para o aumento de eficiência de um time de desenvolvimento e do serviço Web construído}

% Orientador
\advisorgender{masc}
\advisor{Prof.~Dr.}{Fábio Carneiro Mokarzel}{ITA}

% Coorientador
\coadvisorgender{masc}
\coadvisor{Prof.~Dr.}{Inaldo Capistrano Costa}{ITA}

%Coordenador do curso no caso de TG
\bosscoursegender{masc}
\bosscourse{Prof.~Dr.}{Johnny Marques}

% Palavras-Chaves informadas pela Biblioteca -> utilizada na CIP
%\kwcip{Cupim}

% Data da defesa (mês em maiúsculo, se trabalho em inglês, e minúsculo se trabalho em português)
\date{XX}{XXX}{2021}

% Número CDU - (somente para TG)
\cdu{XXX.XX}

% Glossario
\makeglossary
\frontmatter

\begin{document}
% Folha de Rosto e Capa para o caso do TG
\maketitle

% Dedicatoria: Nao esqueca essa secao  ... :-)
\begin{itadedication}
A meus pais, que sempre me apoiaram no caminho que quis seguir, minha irmã
e aos meus amigos, por todo o apoio emocional e moral que me permitiram chegar
ao fim dessa caminhada.
\end{itadedication}

% Agradecimentos
% TODO: Descomentar isso para o TG-2
\begin{itathanks}
A todos os professores que, de uma forma ou de outra, contribuiram para o meu desenvolvimento intelectual
que me permitiu produzir esse trabalho.

À TerraMagna, que abriu espaço para o desenvolvimento desse trabalho durante o meu expediente.

\end{itathanks}

% Epígrafe
\thispagestyle{empty}
\ifhyperref\pdfbookmark[0]{\nameepigraphe}{epigrafe}\fi
\begin{flushright}
\begin{spacing}{1}
\mbox{}\vfill
{\sffamily\itshape
``Doing the same thing repeatedly, and expecting \\
different results is the definition of insanity''\\}
--- \textsc{Albert Einstein}

\end{spacing}
\end{flushright}

% Resumo
\begin{abstract}
\noindent
Este trabalho apresenta um novo modelo de gerador de APIs, com o objetivo de solucionar
problemas que ferramentas atuais apresentam. Esses problemas são relacionados a baixa
integração com outras ferramentas de desenvolvimento, como ORMs e validadores. Mostramos
também que o uso de um gerador que integre eficientemente com o restante do ecossistema
pode resultar em grandes ganhos de eficiência para o time de desenvolvimento, além de
abrir portas para o uso de outras ferramentas que não seriam possíveis sem ele. Uma
motivação para esse trabalho é a busca do aumento de eficiência e automação dentro do
processo de desenvolvimento e a redução na incidência de defeitos em sistemas complexos.

\end{abstract}

% Abstract
\begin{englishabstract}
\noindent
This work presents a new model of API generator, intending to solve problems that
current solutions have. These problems are related to low integration with other
development tools such as ORMs and validators. We also show that the use of a
generator integrating effectively with the rest of the ecosystem can result in
significant efficiency gains for a development team and open doors for the use of
other tools that would not be possible without it. One motivation for this work
is to search for increased efficiency and automation within the development process
and reduce defects in complex systems.

\end{englishabstract}

% Lista de figuras
\listoffigures %opcional

% Lista de tabelas
%\listoftables %opcional

% Lista de abreviaturas
\listofabbreviations
\begin{longtable}{ll}
  API & \textit{Application Programming Interface} \\
  DSL & \textit{Domain Specific Language} \\
  gRPC & \textit{gRPC Remote Procedure Calls} \\
  IDL & \textit{Interface Definition Language} \\
  ORM & \textit{Object-Relational Mapping} \\
  REST & \textit{Representional State Transfer} \\
\end{longtable}

 %opcional

% Lista de simbolos
%\listofsymbol
%\begin{longtable}{ll}
\end{longtable}

 %opcional

% Sumario
\tableofcontents


\mainmatter
% Os capitulos comecam aqui

\chapter{Introdução}
\section{Contexto}

Desenvolvimento de aplicações Web apresenta um variado nível de dificuldade e
complexidade: podem variar desde um simples serviço com poucas operações até
um complexo sistema com dezenas (ou centenas) de serviços interligados de maneira
arbitrária com centenas ou milhares de operações.

Nesse contexto, diversas ferramentas foram desenvolvidas para tentar facilitar
e simplificar o desenvolvimento desses sistemas: geradores de APIs e \textit{object-relational
mapping} (ORMs) são exemplos de ferramentas que caem nessa categoria.

\section{\textit{Interface Definition Languages}}

APIs são normalmente especificadas em \textit{Interface Definition Languages}
(IDLs). Essas linguagens possuem construções que facilitam a construção de uma
especificação, em relação a operações, entradas, saídas e erros. Qual é usada
depende do tipo de API está sendo construida, alguns exemplos são:

\begin{itemize}
\item
  \textit{Web Service Description Language} \cite{wsdl:spec}: usada para
    definir APIs que usam o protocolo SOAP. É uma linguagem baseada em XML.
    Um exemplo de um arquivo WSDL é apresentado em \cref{anex:wsdl-example}.
\item
  \textit{OpenAPI} \cite{openapi:spec}: comumente usada para definir APIs que usam
    o protocolo HTTP, informalmente chamadas de \textit{"Restful APIs"} em referência
    ao conceito de REST definido em \cite{10.5555/932295}. É uma linguagem baseada
    em YAML. um exemplo de um arquivo OpenAPI é apresentado em \cref{anex:openapi-example}.
\item
  \textit{Protocol Buffers} \cite{googl:protobuf}: usada para definir APIs que usam
    o protocolo gRPC \cite{googl:grpc}. O nome se refere tanto a IDL usada para a
    especificação quanto para o formato binário usado para transmitir as mensagens.
    Um exemplo de um arquivo em Protocol Buffers é apresentado em \cref{anex:protobuf-example}.
\end{itemize}

Existem diversas vantagens em usar uma IDL, entre elas:

\begin{enumerate}
\item
  A comunicação entre diferentes times (possivelmente em diferente organizações) que
  precisam interarir via API é mais simples, já que todos os detalhes de interface
  estão especificados em um formato padrão.
\item
  Pelo mesmo motivo do anterior, é muito mais simples construir ferramentas de análise
  sobre a especificação, como geradores de código, validadores, ferramentas de teste,
  etc.
\end{enumerate}

\section{Geradores de APIs}

Geradores de APIs são ferramentas que, via uma especificação de uma API, conseguem
produzir código-fonte em uma linguagem de programação tanto para realizar requisições
a essa API, quanto também conseguem gerar uma base para a implementação da API em si.
O código gerado em ambos os casos contém tanto métodos para as operações suportadas
pela API quanto as estruturas de dados necessárias para interfacear com as mensagens
a serem transmitidas.

Existem diversas vantagens em usar um gerador de API, entre elas:

\begin{enumerate}
\item
  Como toda a parte dos modelos são geradas, tanto para o servidor quanto para o
  cliente, há uma significativa redução no volume de linhas de código-fonte do sistema,
  reduzindo a possibilidade de defeitos \cite{5010260} e facilitando o entendimento do
  projeto por novos desenvolvedores.
\item
  Como o código é gerado a partir da especificação, sabemos que a implementação
  vai estar de acordo com a interface especificada, permitindo que o programador
  foque em implementar a lógica de cada operação, ao invés de se preocupar se
  as estruturas estão de acordo. No caso de consumidores, eles poupam o trabalho
  de ter que implementar um código de integração com a API, que pode conter erros
  e ser difícil de manter, principalmente com relação a mudanças e adições na API.
\item
  O código gerado abstraí toda a camada de comunicação e rede, tanto do servidor
  quanto do cliente, facilitando o entendimento do código que o programador precisa
  implementar.
\end{enumerate}

Existem diversos exemplos de geradores de APIs, alguns são \cite{openapi:gen} e
\cite{googl:protobuf}.

\section{ORMs}

Da mesma forma que IDLs e geradores tentam facilitar o desenvolvimento da
\textit{interface} de uma API, ORMs tentam facilitar a integração do código da
API com o banco de dados usado (seja ele SQL ou NoSQL). Elas são bibliotecas que
abstraem a execução de operações no banco de dados em uma interface amigável para
a linguagem de programação usada. Como são bibliotecas, uma ORM é específica para
a linguagem de programação em que ela foi implementada, e normalmente também é
específica para um banco ou conjunto de bancos.

ORMs normalmente funcionam via anotações e interfaces que o programador precisa
adicionar ou implementar no código-fonte dos modelos. Essas anotações servem para,
por exemplo, mapear o modelo a uma tabela, um campo a uma coluna, ou uma relação
com outro modelo (1-1, 1-muitos ou muitos-muitos).

Um exemplo em Python usando a ORM \texttt{sqlalchemy} é \cref{lst:sqlalchemy-example}.

\begin{listing}[ht]
\begin{minted}{python}
from sqlalchemy import Column, DateTime, String, Integer, ForeignKey, func
from sqlalchemy.orm import relationship, backref
from sqlalchemy.ext.declarative import declarative_base


Base = declarative_base()


class Department(Base):
    __tablename__ = 'department'
    id = Column(Integer, primary_key=True)
    name = Column(String)


class Employee(Base):
    __tablename__ = 'employee'
    id = Column(Integer, primary_key=True)
    name = Column(String)
    # Use default=func.now() to set the default hiring time
    # of an Employee to be the current time when an
    # Employee record was created
    hired_on = Column(DateTime, default=func.now())
    department_id = Column(Integer, ForeignKey('department.id'))
    # Use cascade='delete,all' to propagate the deletion of a Department
    # onto its Employees
    department = relationship(
        Department,
        backref=backref('employees',
                         uselist=True,
                         cascade='delete,all'))

\end{minted}
\caption{Exemplo de código usando \texttt{sqlalchemy}}
\label{lst:sqlalchemy-example}
\end{listing}

ORMs são populares pois simplificam o trabalho do desenvolvedor ao automatizar muitas
operações que são comumente realizadas no banco, e por prover uma DSL caso seja
necessário fazer algo mais complexo. Dependendo da linguagem, essas funcionalidades
podem ser validades em tempo de compilação, previnindo defeitos.

\section{Problema}

Apesar de serem ferramentas muito populares, geradores de APIs e ORMs não integram
bem. O primeiro costuma focar na \textit{interface de comunicação}, não prestando
atenção a detalhes de implementação. Além disso, dado o grande número de ORMs
presente para cada linguagem, e também as diversas formas de se gerar a interface
de comunicação, geradores convencionais não conseguiriam adicionar suporte para
todas as combinações possíveis.

Outro problema em como os geradores são implementados hoje é que eles possuem
suporte limitado a extensões externas ao código-fonte. Dependendo do gerador
utilizado, é necessário implementar um \textit{novo gerador}, o que gera um grande
custo operacional. Alguns exemplos de modificações possíveis:

\begin{itemize}
\item
  Geração automática de testes para as operações \cite{9159071}.
\item
  Validação automática de propriedades das mensagens \cite{envoy:protoc-gen-validate}.
\item
  Integração com ORMs ou outras bibliotecas.
\end{itemize}

Devido a esses problemas, muitas organizações deixam de usar essas ferramentas, os
programadores precisam implementar manualmente códigos que poderiam ser gerados.
Isso aumenta a chance de erros ocorrerem durante a implementação, e ela divergir da
especificação, quanto também diminui a eficiência do time, pois há mais tarefas a
serem realizadas.

Avaliando as tarefas realizadas pelo time de engenharia de uma organização durante o
ano de 2020, foi possível identificar que pelos menos 30\% das tarefas realizadas
eram realicionadas a implementação de modelos, integração com ORMs e com a camada
de comunicação da API. Além disso, desses 30\%, ocorreram diversas vezes tarefas
extras relacionadas com ajustes da implementação para que essa seguisse a especificação.

Na tentativa de solucionar esse problema, esse trabalho propõe um novo modelo de
gerador de APIs, que pode ser extendido para suportar qualquer linguagem ou funcionalidade
nova sem a necessidade de modificar o código-fonte do mesmo.

Esse trabalho é estruturado como se segue. \cref{cap:past-works} faz uma análise
de trabalhos anteriores. \cref{cap:proposal} apresenta, de forma detalhada, o que
foi construído e a metodologia de análise.


\chapter{Trabalhos Anteriores}\label{cap:past-works}
% TODO: maybe talk about LLVM here?

Durante o desenvolvimento desse trabalho, analisamos diversos trabalhos já publicados
relacionados a geradores de APIs, e extensões a IDLs.

\cite{openapi:gen} apresenta uma vasta gama de geradores baseados na IDL OpenAPI. Até
a presente data, apresenta 66 geradores de clientes para 33 linguagens e 41 geradores
de servidores para 16 linguagens. Esse é o principal programa utilizado para gerar
código em projetos que usam OpenAPI para especificação de suas APIs.

Analisando o código-fonte e documentação do projeto, chegamos às seguintes conclusões:

\begin{itemize}
\item
  Os geradores funcionam com base em um modelo genérico baseado em OpenAPI. O processo
  de geração ocorre da seguinte maneira:

  \begin{enumerate}
  \item
    O programa lê a especificação OpenAPI e a transforma no modelo genérico.
  \item
    A classe do gerador modifica esse modelo da forma que precisar, possivelmente
    adicionando propriedades específicas para ele.
  \item
    O programa envia o modelo final para um processador de \textit{templates}, que
    carrega os templates do gerador específico e os renderiza. O resultado dessa
    etapa é o código final.
  \end{enumerate}

  Esse fato pode acabar por limitar a expressividade do gerador, pois o modelo
  OpenAPI não é capaz de expressar, de forma simples, todos os detalhes envolvidos
  em uma linguagem de programação.
\item
  Apesar de ser possível criar um novo gerador customizado, por exemplo, para uma
  linguagem que o projeto não suporte, sem a necessidade de modificar o programa
  em si, os geradores são estruturas monolíticas. Não é possível implementar uma
  funcionalidade nova, e.g. um novo processo de validação, que possa ser usado por
  todos os geradores. Isso limita significativamente a extensibilidade do projeto,
  além de aumentar a carga operacional nessas situações.
\end{itemize}

\cite{googl:protobuf} é o compilador de Protocol Buffers. Ele apresenta uma estrutura
bastante interessante em questão de extensibilidade: o sistema de \textit{plugins}.
Um \textit{plugin} é um programa que recebe uma mensagem \texttt{CodeGeneratorRequest}
como entrada e escreve na saída uma mensagem \texttt{CodeGeneratorResponse}. As
definições dessas duas mensagens são apresentadas em \cref{lst:code-gen-req} e
\cref{lst:code-gen-res}, respectivamente.

\begin{listing}[ht]
\caption{Especificação de \texttt{CodeGeneratorRequest}}
\label{lst:code-gen-req}
\begin{minted}{protobuf}
message CodeGeneratorRequest {
  repeated string file_to_generate = 1;

  optional string parameter = 2;

  repeated FileDescriptorProto proto_file = 15;

  optional Version compiler_version = 3;
}
\end{minted}
\end{listing}

\begin{listing}[ht]
\caption{Especificação de \texttt{CodeGeneratorResponse}}
\label{lst:code-gen-res}
\begin{minted}{protobuf}
message CodeGeneratorResponse {
  optional string error = 1;

  message File {
    optional string name = 1;

    optional string insertion_point = 2;

    optional string content = 15;
  }

  repeated File file = 15;
}
\end{minted}
\end{listing}

Esse sistema é interessante por dois fatores:

\begin{itemize}
\item
  É muito simples adicionar suporte a uma nova linguagem, precisamos apenas
  implementar um \textit{plugin}. Ponto em comum com o trabalho anterior.
\item
  Usando o campo \texttt{file.insertion\_point}, é possível injetar conteúdo
  de um gerador em outro. O segundo gerador pode adicionar esses pontos no
  arquivo gerado por ele, permitindo que outros geradores modifiquem o resultado
  final.

  Isso soluciona, em parte, o problema de adicionar novas funcionalidades em
  um gerador presente em \cite{openapi:gen}. Dois problemas ainda persistem:

  \begin{enumerate}
  \item
    Estamos limitados aos pontos de inserção disponibilizados pelo gerador, o
    que pode variar de uma linguagem para outra. O quão grave é esse problema
    depende do que se quer fazer com o \textit{plugin} e qual é a linguagem objeto.
  \item
    O sistema trabalha em termos de texto (campo \texttt{file.content}). Isso
    limita o quão genérico nossa funcionalidade pode ser, e.g. um \textit{plugin} de
    validação poderia ser genérico caso o resultado fosse mais estruturado.

    Um exemplo de plugin que poderia ser genérico é \cite{envoy:protoc-gen-validate},
    que provê uma extensão para diversas validações. Hoje, ela é limitada para
    as linguagens Go, C++ e Java. Caso o resultado fosse mais estruturado, seria
    possível implementar tal funcionalidade de forma genérica para uma grande
    quantidade de linguagens.
  \end{enumerate}
\end{itemize}

\cite{9159071} se propõe a resolver um problema diferente dos trabalhos anteriores:
ele gera testes baseado em \textit{Property Based Testing} \cite{10.1145/351240.351266}
a partir de especificações OpenAPI que validam que as respostas da API seguem as
propriedades e formatos especificados. O programa é capaz de gerar esses testes
sem a necessidade de nenhuma extensão a especificação.

\cite{sferruzza:hal-01868498} propõe extensões e um gerador para OpenAPI que é
capaz de modelar como uma operação é implementada. O trabalho cria o conceito de
\textit{componentes atômicos e compostos}: componentes atômicos recebem parâmetros
e podem gerar novos valores e componentes compostos fazem a composição de diversos
componentes para definir uma dada lógica. O programa então é capaz de validar que
as definições e uso dos componentes são válidas, tanto em questão de todas as
variáveis estarem disponíveis quanto que os tipos estão corretos. O gerador é capaz
de gerar código que define esses componentes.

Por fim, \cite{r2c:semgrep} é uma ferramenta de análise estática que suporta uma
quantidade impressionante de linguagens. O diferencial dessa ferramenta é que
o usuário pode criar novas regras de análise utilizando uma DSL implementada sobre
YAML, permitindo que a mesma regra seja aplicada em diversas linguagens. Isso
é possível pois todas as linguagens que ela suporta são convertidas para
\textit{uma mesma linguagem intermediária} antes das regras serem aplicadas. Essa
funcionalidade é similar a situação em que queremos implementar uma funcionalidade
no nosso gerador de forma independente da linguagem objeto.


\chapter{Proposta}\label{cap:proposal}
A hipótese base desse trabalho é que, baseado nos projetos mencionados anteriormente,
é possível implementar um novo modelo de gerador que seja muito mais extensível, amigável
e que consiga implementar funcionalidades mais genéricas. Além disso, outra hipótese
é que, com base nessa extensibilidade, conseguimos aumentar de forma considerável a
eficiência de um time de desenvolvimento, utilizando o gerador para automatizar
trabalhos repetitivos.

O objetivo desse trabalho então é implementar esse novo modelo, além de uma série de
funcionalidades a partir do sistema de extensibilidade. Para validar que atingimos nosso
objetivo, implementaremos uma nova API em nível prova de conceito, utilizando esse gerador,
e verificaremos o que seria necessário a mais para a implementação, caso não tivessemos
acesso a tal ferramenta.

A \cref{tbl:cronograma} apresenta uma estimativa das etapas a serem realizadas durante
o desenvolvimento do trabalho.

\begin{table}[ht]
\centering
\begin{tabular}{c | c}
Passo & Período Estimado \\
\hline
\textit{Parser Protocol Buffers} e Linguagem Intermediária & Maio \\
Análises Semânticas sobre a LI & Junho \\
Implementação da interface de \textit{plugins} & Julho \\
\textit{Plugin} para HTTP & Inicio Agosto \\
\textit{Plugin} para ORM & Agosto/Setembro \\
Gerador para linguagem-objeto escolhida & Final Setembro \\
Implementação da nova API em POC & Inicio Outubro \\
Análise dos resultados & Final de Outubro
\end{tabular}
\caption{Cronograma de atividades}
\label{tbl:cronograma}
\end{table}


Para validar o quanto a ferramenta auxilia no desenvolvimento, iremos desenvolver um
serviço simples de um \textit{micro-blog}, com usuários, postagens e curtidas utilizando
o que foi desenvolvido. A partir dessa implementação, analisaremos a quantidade de
\textit{trabalho extra} que seria necessário, caso não tivessemos acesso a um gerador
de código integrado. Com isso, saberemos o ganho de eficiência dado pela ferramenta.




% REFERENCIAS BIBLIOGRAFICAS
\renewcommand\bibname{\itareferencesnamebabel} %renomear título do capítulo referências
\bibliography{referencias}

\annex
\chapter{Exemplo em WSDL}\label{anex:wsdl-example}
\begin{minted}{xml}
<?xml version="1.0" encoding="UTF-8"?>
<description xmlns="http://www.w3.org/ns/wsdl"
             xmlns:tns="http://www.tmsws.com/wsdl20sample"
             xmlns:whttp="http://schemas.xmlsoap.org/wsdl/http/"
             xmlns:wsoap="http://schemas.xmlsoap.org/wsdl/soap/"
             targetNamespace="http://www.tmsws.com/wsdl20sample">

<documentation>
  This is a sample WSDL 2.0 document.
</documentation>

<!-- Abstract type -->
  <types>
    <xs:schema xmlns:xs="http://www.w3.org/2001/XMLSchema"
               xmlns="http://www.tmsws.com/wsdl20sample"
               targetNamespace="http://www.example.com/wsdl20sample">
     <xs:element name="Error" type="Error"/>
     <xs:element name="Pet" type="Pet"/>
     <xs:element name="Pets" type="Pets"/>
     <xs:element name="NewPet" type="NewPet"/>

     <xs:element name="ListPetsRequest" type="ListPetsRequest"/>
     <xs:element name="ShowPetByIdRequest" type="ShowPetByIdRequest"/>

     <xs:complexType name="Error">
      <xs:attribute name="code" type="xs:int"/>
      <xs:attribute name="message" type="xs:string"/>
     </xs:complexType>
     <xs:complexType name="Pet">
      <xs:attribute name="id" type="xs:int"/>
      <xs:attribute name="name" type="xs:string"/>
      <xs:attribute name="tag" type="xs:string" nillable="true"/>
     </xs:complexType>
     <xs:complexType name="NewPet">
      <xs:attribute name="name" type="xs:string"/>
      <xs:attribute name="tag" type="xs:string" nillable="true"/>
     </xs:complexType>
     <xs:complexType name="Pets">
      <xs:sequence>
        <xs:element minOccurs="0" name="pets" type="tns:Pet"/>
      </xs:sequence>
     </xs:complexType>

     <xs:complexType name="ListPetsRequest">
      <xs:attribute name="limit" type="xs:int" nillable="true"/>
      <xs:attribute name="cursor" type="xs:string" nillable="true"/>
     </xs:complexType>

     <xs:complexType name="ShowPetByIdRequest">
      <xs:attribute name="id" type="xs:int"/>
     </xs:complexType>
    </xs:schema>
  </types>

<!-- Abstract interfaces -->
  <interface name="PetStoreInterface">
    <fault name="Error1" element="tns:Error"/>
    <operation name="ListPets" pattern="http://www.w3.org/ns/wsdl/in-out">
      <input messageLabel="In" element="tns:ListPetsRequest"/>
      <output messageLabel="Out" element="tns:Pets"/>
    </operation>
    <operation name="CreatePet" pattern="http://www.w3.org/ns/wsdl/in-out">
      <input messageLabel="In" element="tns:NewPet"/>
      <output messageLabel="Out" element="tns:Pet"/>
    </operation>
    <operation name="ShowPetById" pattern="http://www.w3.org/ns/wsdl/in-out">
      <input messageLabel="In" element="tns:ShowPetByIdRequest"/>
      <output messageLabel="Out" element="tns:Pet"/>
    </operation>
  </interface>

<!-- Concrete Binding Over HTTP -->
  <binding name="HttpBinding" interface="tns:PetStoreInterface"
           type="http://www.w3.org/ns/wsdl/http">
    <operation ref="tns:ListPets" whttp:method="GET"/>
    <operation ref="tns:CreatePet" whttp:method="POST"/>
    <operation ref="tns:ShowPetById" whttp:method="GET"/>
  </binding>

<!-- Concrete Binding with SOAP-->
  <binding name="SoapBinding" interface="tns:PetStoreInterface"
           type="http://www.w3.org/ns/wsdl/soap"
           wsoap:protocol="http://www.w3.org/2003/05/soap/bindings/HTTP/"
           wsoap:mepDefault="http://www.w3.org/2003/05/soap/mep/request-response">
    <operation ref="tns:Get" />
    <operation ref="tns:CreatePet" />
    <operation ref="tns:ShowPetById" />
  </binding>

<!-- Web Service offering endpoints for both bindings-->
  <service name="PetStoreService" interface="tns:PetStoreInterface">
    <endpoint name="HttpEndpoint"
              binding="tns:HttpBinding"
              address="http://www.example.com/rest/"/>
    <endpoint name="SoapEndpoint"
              binding="tns:SoapBinding"
              address="http://www.example.com/soap/"/>
  </service>
</description>
\end{minted}


\chapter{Exemplo em OpenAPI}\label{anex:openapi-example}
\input{Anexos/anexoB.tex}

\chapter{Exemplo em Protocol Buffers}\label{anex:protobuf-example}
\begin{minted}{protobuf}
syntax = "proto3";

import "google/protobuf/empty.proto";

package petstore;

// Interface exported by the server.
service PetStoreService {
  // List all pets.
  rpc ListPets(ListPetsRequest) retuns (ListPetsResponse);

  // Create a pet.
  rpc CreatePet(CreatePetRequest) returns (google.protobuf.Empty);

  // Info for a specific pet.
  rpc ShowPetById(ShowPetByIdRequest) returns (Pet);
}

message ListPetsRequest {
  // How many items to return at one time (max 100).
  optional uint32 limit = 1;
  // A link to the page of responses
  optional string cursor = 2;
}

message ListPetsResponse {
  // A paged array of pets
  repeated Pet pets = 1;
  // A link to the next page of responses
  optional string cursor = 2;
}

message CreatePetRequest {
  string name = 2;
  optional string tag = 3;
}

message ShowPetByIdRequest {
  // The id of the pet to retrieve
  uint64 id = 1;
}

message Pet {
  uint64 id = 1;
  string name = 2;
  optional string tag = 3;
}
\end{minted}



% Glossario
%\itaglossary
%\printglossary

% Folha de Registro do Documento
% Valores dos campos do formulario
\FRDitadata{25 de março de 2015}
\FRDitadocnro{DCTA/ITA/DM-018/2015} %(o número de registro você solicita a biblioteca)
\FRDitaorgaointerno{Instituto Tecnológico de Aeronáutica -- ITA}
%Exemplo no caso de pós-graduação: Instituto Tecnol{\'o}gico de Aeron{\'a}utica -- ITA
\FRDitapalavrasautor{Cupim; Cimento; Estruturas}
\FRDitapalavrasresult{Cupim; Dilema; Construção}
%Exemplo no caso de graduação (TG):
\FRDitapalavraapresentacao{Trabalho de Graduação, ITA, São José dos Campos, 2015. \NumPenultimaPagina\ páginas.}
%Exemplo no caso de pós-graduação (msc, dsc):
%\FRDitapalavraapresentacao{ITA, São José dos Campos. Curso de Mestrado. Programa de Pós-Graduação em Engenharia Aeronáutica e Mecânica. Área de Sistemas Aeroespaciais e Mecatrônica. Orientador: Prof.~Dr. Adalberto Santos Dupont. Coorientadora: Prof$^\textnormal{a}$.~Dr$^\textnormal{a}$. Doralice Serra. Defesa em 05/03/2015. Publicada em 25/03/2015.}
\FRDitaresumo{Este trabalho apresenta um novo modelo de gerador de APIs, com o objetivo de solucionar
problemas que ferramentas atuais apresentam. Esses problemas são relacionados a baixa
integração com outras ferramentas de desenvolvimento, como ORMs e validadores. Mostramos
também que o uso de um gerador que integre eficientemente com o restante do ecossistema
pode resultar em grandes ganhos de eficiência para o time de desenvolvimento, além de
abrir portas para o uso de outras ferramentas que não seriam possíveis sem ele. Uma
motivação para esse trabalho é a busca do aumento de eficiência e automação dentro do
processo de desenvolvimento e a redução na incidência de defeitos em sistemas complexos.
}
%  Primeiro Parametro: Nacional ou Internacional -- N/I
%  Segundo parametro: Ostensivo, Reservado, Confidencial ou Secreto -- O/R/C/S
\FRDitaOpcoes{N}{O}
% Cria o formulario
%\itaFRD


\end{document}

% Fim do Documento. O massacre acabou!!! :-)
