%%% Exemplo de utilização da classe ITA
%%%
%%%   por        Fábio Fagundes Silveira   -  ffs [at] ita [dot] br
%%%              Benedito C. O. Maciel     -  bcmaciel [at] ita [dot] br
%%%              Giovani Volnei Meinertz   -  giovani [at] ita [dot] br
%%%    	         Hudson Alberto Bode       -  bode [at] ita [dot]br
%%%    	         P. I. Braga de Queiroz    -  pi [at] ita [dot] br
%%%    	         Jorge A. B. Gripp         -  gripp [at] ita [dot] br
%%%    	         Juliano Monte-Mor         -  jamontemor [at] yahoo [dot] com [dot] br
%%%    	         Tarcisio A. B. Gripp      -  tarcisio.gripp [at] gmail [dot] com
%%%
%%%   Versão para overleaf:
%%%   por           Alejandro A. Rios Cruz - aarc.88@gmail.com
%%%                 Saulo Gómez            - sagomezs@unal.edu.co
%%%  IMPORTANTE: O texto contido neste exemplo nao significa absolutamente nada.  :-)
%%%              O intuito aqui eh demonstrar os comandos criados na classe e suas
%%%              respectivas utilizacoes.
%%%
%%%  Tese.tex  2016-08-25
%%%  $HeadURL: http://www.apgita.org.br/apgita/teses-e-latex.php $
%%%
%%% ITALUS
%%% Instituto Tecnológico de Aeronáutica --- ITA, Sao Jose dos Campos, Brasil
%%%                   http://groups.yahoo.com/group/italus/
%%% Discussion list: italus {at} yahoogroups.com
%%%
%++++++++++++++++++++++++++++++++++++++++++++++++++++++++++++++++++++++++++++++
% Para alterar o TIPO DE DOCUMENTO, preencher a linha abaixo \documentclass[?]{?}
%   \documentclass[tg]{ita}			= Trabalho de Graduacao
%   \documentclass[tgfem]{ita}	= Para Engenheiras
%   								msc     		= Dissertacao de Mestrado
%   								mscfem   		= Para Mestras
%   								dsc      		= Tese de Doutorado
%   								dscfem   		= Para Doutoras
%   								quali    		= Exame de Qualificacao
%   								qualifem 		= Exame de Qualificacao para Doutoras
% Para 'Draft Version'/'Versao Preliminar' com data no rodape, adicionar 'dv':
%   \documentclass[dsc, dv]{ita}
% Para trabalhos em Inglês, adicionar 'eng':
%   \documentclass[dsc, eng]{ita}
%		\documentclass[dsc, eng, dv]{ita}
%++++++++++++++++++++++++++++++++++++++++++++++++++++++++++++++++++++++++++++++
\documentclass[tg]{ita}    % ITA.cls based on standard book.cls
% Quando alterar a classe, por exemplo de [msc] para [msc, eng]) rode mais uma vez o botão BUILD OUTPUT caso haja erro
\usepackage{ae}
\usepackage{graphicx}
\usepackage{epsfig}
\usepackage{amsmath}
\usepackage{amssymb}
\usepackage{subfig}
\usepackage{multirow}
\usepackage{float}
\usepackage{amsthm}
\usepackage{url}         % formats URL addresses properly
\usepackage{appendix}    % allows appendix section to be included
\usepackage{lscape}      % allows a page to be rendered in landscape mode
\usepackage{multicol}    % allows text in multi columns
\usepackage{cancel}      % needed to show canceled terms in equations
\usepackage{lettrine}
\usepackage{float}
\usepackage{placeins}
\usepackage[outputdir=latex.out]{minted}

\usemintedstyle{friendly}


%HHHHHHHHHHHHHHHHHHHHHHHHHHHHHHHHHHHHHHHHHHHHHHHHHHHHHHHHHHHHHHHHHHHHHHHHHHHHHHHHHHHHHHHHHHHHHHHHHHHHHHHHHHHH
%\usepackage{subfigure}
%\usepackage{subfigmat}
%PACOTEFIGURAS_SE _ERRADO_ESXCLUIR_ACIMA
\usepackage{booktabs}
%PACOTETABELAS_SE _ERRADO_ESXCLUIR_ACIMA
%HHHHHHHHHHHHHHHHHHHHHHHHHHHHHHHHHHHHHHHHHHHHHHHHHHHHHHHHHHHHHHHHHHHHHHHHHHHHHHHHHHHHHHHHHHHHHHHHHHHHHHHHHHHH

%++++++++++++++++++++++++++++++++++++++++++++++++++++++++++++++++++++++++++++++
% Espaçamento padrão de todo o documento
%++++++++++++++++++++++++++++++++++++++++++++++++++++++++++++++++++++++++++++++
\onehalfspacing

%singlespacing Para um espaçamento simples
%onehalfspacing Para um espaçamento de 1,5
%doublespacing Para um espaçamento duplo

%++++++++++++++++++++++++++++++++++++++++++++++++++++++++++++++++++++++++++++++
% Identificacoes (se o trabalho for em inglês, insira os dados em inglês)
% Para entradas abreviadas de Professora (Profa.) em português escreva: Prof$^\textnormal{a}$.
%++++++++++++++++++++++++++++++++++++++++++++++++++++++++++++++++++++++++++++++
\course{Engenheria da Computação}

% Autor do trabalho: Nome Sobrenome
\authorgender{masc}
\author{Luis Cláudio Magalhães de}{Holanda}
\itaauthoraddress{Rua Engenheiro Prudente Merieles de Morais}{12.243-750}{São José dos Campos--SP}

% Titulo da Tese/Dissertação
\title{Aplicação de Model Driven Development para o aumento de eficiência de um time de desenvolvimento e do serviço Web construído}

% Orientador
\advisorgender{masc}
\advisor{Prof.~Dr.}{Fábio Carneiro Mokarzel}{ITA}

% Coorientador
\coadvisorgender{masc}
\coadvisor{Prof.~Dr.}{Inaldo Capistrano Costa}{ITA}

%Coordenador do curso no caso de TG
\bosscoursegender{masc}
\bosscourse{Prof.~Dr.}{Johnny Marques}

% Palavras-Chaves informadas pela Biblioteca -> utilizada na CIP
\kwcip{Cupim}
\kwcip{Dilema}
\kwcip{Construção}

% membros da banca examinadora

\examiner{Prof. Dr.}{Alan Turing}{Presidente}{ITA}
\examiner{Prof. Dr.}{Linus Torwald}{}{UXXX}
\examiner{Prof. Dr.}{Richard Stallman}{}{UYYY}
\examiner{Prof. Dr.}{Donald Duck}{}{DYSNEY}
\examiner{Prof. Dr.}{Mickey Mouse}{}{DISNEY}

% Data da defesa (mês em maiúsculo, se trabalho em inglês, e minúsculo se trabalho em português)
\date{5}{março}{2021}

% Número CDU - (somente para TG)
\cdu{XXX.XX}

% Glossario
\makeglossary
\frontmatter

\begin{document}
% Folha de Rosto e Capa para o caso do TG
\maketitle

% Dedicatoria: Nao esqueca essa secao  ... :-)
\begin{itadedication}
A meus pais, que sempre me apoiaram no caminho que quis seguir, minha irmã
e aos meus amigos, por todo o apoio emocional e moral que me permitiram chegar
ao fim dessa caminhada.
\end{itadedication}

% Agradecimentos
% TODO: Descomentar isso para o TG-2
\begin{itathanks}
A todos os professores que, de uma forma ou de outra, contribuiram para o meu desenvolvimento intelectual
que me permitiu produzir esse trabalho.

À TerraMagna, que abriu espaço para o desenvolvimento desse trabalho durante o meu expediente.

\end{itathanks}

% Epígrafe
\thispagestyle{empty}
\ifhyperref\pdfbookmark[0]{\nameepigraphe}{epigrafe}\fi
\begin{flushright}
\begin{spacing}{1}
\mbox{}\vfill
{\sffamily\itshape
``Doing the same thing repeatedly, and expecting \\
different results is the definition of insanity''\\}
--- \textsc{Albert Einstein}

\end{spacing}
\end{flushright}

% Resumo
\begin{abstract}
\noindent
Este trabalho apresenta um novo modelo de gerador de APIs, com o objetivo de solucionar
problemas que ferramentas atuais apresentam. Esses problemas são relacionados a baixa
integração com outras ferramentas de desenvolvimento, como ORMs e validadores. Mostramos
também que o uso de um gerador que integre eficientemente com o restante do ecossistema
pode resultar em grandes ganhos de eficiência para o time de desenvolvimento, além de
abrir portas para o uso de outras ferramentas que não seriam possíveis sem ele. Uma
motivação para esse trabalho é a busca do aumento de eficiência e automação dentro do
processo de desenvolvimento e a redução na incidência de defeitos em sistemas complexos.

\end{abstract}

% Abstract
\begin{englishabstract}
\noindent
This work presents a new model of API generator, intending to solve problems that
current solutions have. These problems are related to low integration with other
development tools such as ORMs and validators. We also show that the use of a
generator integrating effectively with the rest of the ecosystem can result in
significant efficiency gains for a development team and open doors for the use of
other tools that would not be possible without it. One motivation for this work
is to search for increased efficiency and automation within the development process
and reduce defects in complex systems.

\end{englishabstract}

% Lista de figuras
\listoffigures %opcional

% Lista de tabelas
%\listoftables %opcional

% Lista de abreviaturas
\listofabbreviations
\begin{longtable}{ll}
  API & \textit{Application Programming Interface} \\
  DSL & \textit{Domain Specific Language} \\
  gRPC & \textit{gRPC Remote Procedure Calls} \\
  IDL & \textit{Interface Definition Language} \\
  ORM & \textit{Object-Relational Mapping} \\
  REST & \textit{Representional State Transfer} \\
\end{longtable}

 %opcional

% Lista de simbolos
%\listofsymbol
%\begin{longtable}{ll}
\end{longtable}

 %opcional

% Sumario
\tableofcontents


\mainmatter
% Os capitulos comecam aqui

\chapter{Introdução}
\section{Contexto}

Desenvolvimento de aplicações Web apresenta um variado nível de dificuldade e
complexidade: podem variar desde um simples serviço com poucas operações até
um complexo sistema com dezenas (ou centenas) de serviços interligados de maneira
arbitrária com centenas ou milhares de operações.

% TODO: Introduzir o conceito de API

Nesse contexto, diversas ferramentas foram desenvolvidas para tentar facilitar
e simplificar o desenvolvimento desses sistemas: geradores de APIs e \textit{object-relational
mapping} (ORMs) são exemplos de ferramentas que caem nessa categoria.

\section{\textit{Interface Definition Languages}}

APIs são normalmente especificadas em \textit{Interface Definition Languages}
(IDLs). Essas linguagens possuem construções que facilitam a construção de uma
especificação, em relação a operações, entradas, saídas e erros. Qual é usada
depende do tipo de API está sendo construida, alguns exemplos são:

\begin{itemize}
\item
  \textit{Web Service Description Language} (WSDL - \cite{wsdl:spec}): usada para
    definir APIs que usam o protocolo SOAP. É uma linguagem baseada em XML.
    Um exemplo de um arquivo WSDL é apresentado em \ref{anex:wsdl-example}.
\item
  \textit{OpenAPI} - \cite{openapi:spec}: comumente usada para definir APIs que usam
    o protocolo HTTP, informalmente chamadas de "APIs Restful" em referência ao conceito
    de REST criado por Roy Fielding em \cite{10.5555/932295}. É uma linguagem baseada
    em YAML. um exemplo de um arquivo OpenAPI é apresentado em \ref{anex:openapi-example}.
\item
  \textit{Protocol Buffers} - \cite{googl:protobuf}: usada para definir APIs que usam
    o protocolo gRPC. O nome se refere tanto a IDL usada para a especificação quanto
    para o formato binário usado para transmitir as mensagens.
\end{itemize}

Uma grande vantagem de criar uma espeficação para a sua API usando uma IDL é uma
comunicação mais simples com quem quer que deseje a consumir, pois tudo que o cliente
precisa saber está disponível nessa especificação.

Outra vantagem é que, por ser um formato padronizado, é possível criar ferramentas
de análise sobre a especificação, como geradores e validadores.

\section{Geradores de APIs}

Geradores de APIs são ferramentas que, via uma especificação de uma API,
conseguem produzir código-fonte em uma linguagem de programação tanto para
realizar requisições a essa API, quanto também conseguem gerar uma base para
a implementação da API em si. O código gerado em ambos os casos contém tanto
métodos para as operações suportadas pela API quanto as estruturas de dados
necessárias para interfacear com as mensagens a serem transmitidas.

\section{ORMs}




% REFERENCIAS BIBLIOGRAFICAS
\renewcommand\bibname{\itareferencesnamebabel} %renomear título do capítulo referências
\bibliography{referencias}

\annex
\chapter{Exemplo em WSDL}\label{anex:wsdl-example}
\begin{minted}{xml}
<?xml version="1.0" encoding="UTF-8"?>
<description xmlns="http://www.w3.org/ns/wsdl"
             xmlns:tns="http://www.tmsws.com/wsdl20sample"
             xmlns:whttp="http://schemas.xmlsoap.org/wsdl/http/"
             xmlns:wsoap="http://schemas.xmlsoap.org/wsdl/soap/"
             targetNamespace="http://www.tmsws.com/wsdl20sample">

<documentation>
  This is a sample WSDL 2.0 document.
</documentation>

<!-- Abstract type -->
  <types>
    <xs:schema xmlns:xs="http://www.w3.org/2001/XMLSchema"
               xmlns="http://www.tmsws.com/wsdl20sample"
               targetNamespace="http://www.example.com/wsdl20sample">
     <xs:element name="Error" type="Error"/>
     <xs:element name="Pet" type="Pet"/>
     <xs:element name="Pets" type="Pets"/>
     <xs:element name="NewPet" type="NewPet"/>

     <xs:element name="ListPetsRequest" type="ListPetsRequest"/>
     <xs:element name="ShowPetByIdRequest" type="ShowPetByIdRequest"/>

     <xs:complexType name="Error">
      <xs:attribute name="code" type="xs:int"/>
      <xs:attribute name="message" type="xs:string"/>
     </xs:complexType>
     <xs:complexType name="Pet">
      <xs:attribute name="id" type="xs:int"/>
      <xs:attribute name="name" type="xs:string"/>
      <xs:attribute name="tag" type="xs:string" nillable="true"/>
     </xs:complexType>
     <xs:complexType name="NewPet">
      <xs:attribute name="name" type="xs:string"/>
      <xs:attribute name="tag" type="xs:string" nillable="true"/>
     </xs:complexType>
     <xs:complexType name="Pets">
      <xs:sequence>
        <xs:element minOccurs="0" name="pets" type="tns:Pet"/>
      </xs:sequence>
     </xs:complexType>

     <xs:complexType name="ListPetsRequest">
      <xs:attribute name="limit" type="xs:int" nillable="true"/>
      <xs:attribute name="cursor" type="xs:string" nillable="true"/>
     </xs:complexType>

     <xs:complexType name="ShowPetByIdRequest">
      <xs:attribute name="id" type="xs:int"/>
     </xs:complexType>
    </xs:schema>
  </types>

<!-- Abstract interfaces -->
  <interface name="PetStoreInterface">
    <fault name="Error1" element="tns:Error"/>
    <operation name="ListPets" pattern="http://www.w3.org/ns/wsdl/in-out">
      <input messageLabel="In" element="tns:ListPetsRequest"/>
      <output messageLabel="Out" element="tns:Pets"/>
    </operation>
    <operation name="CreatePet" pattern="http://www.w3.org/ns/wsdl/in-out">
      <input messageLabel="In" element="tns:NewPet"/>
      <output messageLabel="Out" element="tns:Pet"/>
    </operation>
    <operation name="ShowPetById" pattern="http://www.w3.org/ns/wsdl/in-out">
      <input messageLabel="In" element="tns:ShowPetByIdRequest"/>
      <output messageLabel="Out" element="tns:Pet"/>
    </operation>
  </interface>

<!-- Concrete Binding Over HTTP -->
  <binding name="HttpBinding" interface="tns:PetStoreInterface"
           type="http://www.w3.org/ns/wsdl/http">
    <operation ref="tns:ListPets" whttp:method="GET"/>
    <operation ref="tns:CreatePet" whttp:method="POST"/>
    <operation ref="tns:ShowPetById" whttp:method="GET"/>
  </binding>

<!-- Concrete Binding with SOAP-->
  <binding name="SoapBinding" interface="tns:PetStoreInterface"
           type="http://www.w3.org/ns/wsdl/soap"
           wsoap:protocol="http://www.w3.org/2003/05/soap/bindings/HTTP/"
           wsoap:mepDefault="http://www.w3.org/2003/05/soap/mep/request-response">
    <operation ref="tns:Get" />
    <operation ref="tns:CreatePet" />
    <operation ref="tns:ShowPetById" />
  </binding>

<!-- Web Service offering endpoints for both bindings-->
  <service name="PetStoreService" interface="tns:PetStoreInterface">
    <endpoint name="HttpEndpoint"
              binding="tns:HttpBinding"
              address="http://www.example.com/rest/"/>
    <endpoint name="SoapEndpoint"
              binding="tns:SoapBinding"
              address="http://www.example.com/soap/"/>
  </service>
</description>
\end{minted}

\chapter{Exemplo em OpenAPI}\label{anex:openapi-example}
\input{Anexos/anexoB.tex}

% Glossario
%\itaglossary
%\printglossary

% Folha de Registro do Documento
% Valores dos campos do formulario
\FRDitadata{25 de março de 2015}
\FRDitadocnro{DCTA/ITA/DM-018/2015} %(o número de registro você solicita a biblioteca)
\FRDitaorgaointerno{Instituto Tecnológico de Aeronáutica -- ITA}
%Exemplo no caso de pós-graduação: Instituto Tecnol{\'o}gico de Aeron{\'a}utica -- ITA
\FRDitapalavrasautor{Cupim; Cimento; Estruturas}
\FRDitapalavrasresult{Cupim; Dilema; Construção}
%Exemplo no caso de graduação (TG):
\FRDitapalavraapresentacao{Trabalho de Graduação, ITA, São José dos Campos, 2015. \NumPenultimaPagina\ páginas.}
%Exemplo no caso de pós-graduação (msc, dsc):
%\FRDitapalavraapresentacao{ITA, São José dos Campos. Curso de Mestrado. Programa de Pós-Graduação em Engenharia Aeronáutica e Mecânica. Área de Sistemas Aeroespaciais e Mecatrônica. Orientador: Prof.~Dr. Adalberto Santos Dupont. Coorientadora: Prof$^\textnormal{a}$.~Dr$^\textnormal{a}$. Doralice Serra. Defesa em 05/03/2015. Publicada em 25/03/2015.}
\FRDitaresumo{Este trabalho apresenta um novo modelo de gerador de APIs, com o objetivo de solucionar
problemas que ferramentas atuais apresentam. Esses problemas são relacionados a baixa
integração com outras ferramentas de desenvolvimento, como ORMs e validadores. Mostramos
também que o uso de um gerador que integre eficientemente com o restante do ecossistema
pode resultar em grandes ganhos de eficiência para o time de desenvolvimento, além de
abrir portas para o uso de outras ferramentas que não seriam possíveis sem ele. Uma
motivação para esse trabalho é a busca do aumento de eficiência e automação dentro do
processo de desenvolvimento e a redução na incidência de defeitos em sistemas complexos.
}
%  Primeiro Parametro: Nacional ou Internacional -- N/I
%  Segundo parametro: Ostensivo, Reservado, Confidencial ou Secreto -- O/R/C/S
\FRDitaOpcoes{N}{O}
% Cria o formulario
%\itaFRD

\end{document}
% Fim do Documento. O massacre acabou!!! :-)
