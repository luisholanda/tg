\section{Contexto}

Desenvolvimento de aplicações Web apresenta um variado nível de dificuldade e
complexidade: podem variar desde um simples serviço com poucas operações até
um complexo sistema com dezenas (ou centenas) de serviços interligados de maneira
arbitrária com centenas ou milhares de operações.

% TODO: Introduzir o conceito de API

Nesse contexto, diversas ferramentas foram desenvolvidas para tentar facilitar
e simplificar o desenvolvimento desses sistemas: geradores de APIs e \textit{object-relational
mapping} (ORMs) são exemplos de ferramentas que caem nessa categoria.

\section{\textit{Interface Definition Languages}}

APIs são normalmente especificadas em \textit{Interface Definition Languages}
(IDLs). Essas linguagens possuem construções que facilitam a construção de uma
especificação, em relação a operações, entradas, saídas e erros. Qual é usada
depende do tipo de API está sendo construida, alguns exemplos são:

\begin{itemize}
\item
  \textit{Web Service Description Language} (WSDL - \cite{wsdl:spec}): usada para
    definir APIs que usam o protocolo SOAP. É uma linguagem baseada em XML.
    Um exemplo de um arquivo WSDL é apresentado em \ref{anex:wsdl-example}.
\item
  \textit{OpenAPI} - \cite{openapi:spec}: comumente usada para definir APIs que usam
    o protocolo HTTP, informalmente chamadas de "APIs Restful" em referência ao conceito
    de REST criado por Roy Fielding em \cite{10.5555/932295}. É uma linguagem baseada
    em YAML. um exemplo de um arquivo OpenAPI é apresentado em \ref{anex:openapi-example}.
\item
  \textit{Protocol Buffers} - \cite{googl:protobuf}: usada para definir APIs que usam
    o protocolo gRPC. O nome se refere tanto a IDL usada para a especificação quanto
    para o formato binário usado para transmitir as mensagens.
\end{itemize}

Uma grande vantagem de criar uma espeficação para a sua API usando uma IDL é uma
comunicação mais simples com quem quer que deseje a consumir, pois tudo que o cliente
precisa saber está disponível nessa especificação.

Outra vantagem é que, por ser um formato padronizado, é possível criar ferramentas
de análise sobre a especificação, como geradores e validadores.

\section{Geradores de APIs}

Geradores de APIs são ferramentas que, via uma especificação de uma API,
conseguem produzir código-fonte em uma linguagem de programação tanto para
realizar requisições a essa API, quanto também conseguem gerar uma base para
a implementação da API em si. O código gerado em ambos os casos contém tanto
métodos para as operações suportadas pela API quanto as estruturas de dados
necessárias para interfacear com as mensagens a serem transmitidas.

\section{ORMs}

